\subsection{サーバ班}
\par
ここでは,Cool Japanimationサーバ班の活動内容について述べる.以下に活動内容の詳細を示す.
\par
・サーバ班の技術習得
\par
前期は,サーバ開発に必要な技術習得とアプリケーションに最適なサーバの調査を行った.具体的には,アドバイザである新見先生に
相談し,OSSセミナーの過去資料を頂き,その資料を元にサーバの学習を行った.アプリケーション決定後は,Cool Japanimationの
サーバに最適なデータベースサーバを選択し,CakePHPやサーバ構築の学習を行った.
\par
・アプリケーションの実装
\par
後期は,通信方法の調査と決定したアプリケーションをもとに,必要な機能とデータの洗い出しをおこなった.
開発した各プラットフォームのアプリケーションとの通信方法について,通信問題が発生したり,知識が不足をしていたため
調査をおこなった.それと並行して必要なデータを確定,データベースの設計と構築をおこなった.次に,PHPを用いアプリケーションと
通信をおこないデータベースを操作するファイルを作成した.作成したファイルは,機能が多いアプリケーションに加え,
各プラットフォームごとに通信の方法が異なり,全てのメンバと情報共有をすることが困難であった.そこで,サーバ班は
すべてのメンバと通信に関する情報を共有するためにLINEでの会議をおこなった.この会議では,現在どのような機能があり,
どのプラットフォームとはどのように通信をするのか,現在どこまで実装できているのか,何か問題が発生していないのか
という情報の共有をおこなった.PHP ファイルを完成させた後は,他プラットフォームメンバと相談し,サーバ班は実装された機能と
実際に通信ができているのか通信テストをおこなった.その後は,デモ用のデータの作成などをおこなった.



\bunseki{木津 智博(未来大)}
