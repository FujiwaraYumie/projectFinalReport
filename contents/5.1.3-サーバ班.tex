\subsection{サーバ班}
\par
ここでは,Cool Japanimationサーバ班の活動内容について述べる.以下に活動内容の詳細を示す.
\par
・サーバ班の技術習得
\par
前期は,サーバ開発に必要な技術習得とアプリケーションに最適なサーバの調査を行った.具体的には,アドバイザである新見先生に
相談し,OSSセミナーの過去資料を頂き,その資料を元にサーバの学習をした.アプリケーション決定後は,Cool Japanimationの
サーバに最適なデータベースサーバを選択し,CakePHPやサーバ構築,LinkedOpenData(以下LOD)の学習を行った.
LODとは,Web上でコンピュータ処理に適したデータを公開・共有するための技術の総称で,今回はアニメ情報を集めるために
Fusekiというサーバを構築し,活用する.また,Cool Japanimationは未来大と長崎大のサーバ班が担当するため,サーバ担当の
住み分けを行い,主なプログラムの開発とデータベースの用意を長崎大,サーバの構築と設定ファイルの変更,データベース作成の
補助を未来大と決め,問題発生時の対処法は,両大学のサーバ班が全員で会議を開き,現状と発生した問題について情報共有を行い,
全員で対処するということに決定した.
\par
・アプリケーションの実装
\par
後期は,通信方法の調査と決定したアプリケーションをもとに,必要な機能とデータの洗い出しを行った.
開発した各プラットフォームのアプリケーションとの通信方法について,通信問題が発生したり,知識が不足をしていたため
調査を行った.それと並行して必要なデータを確定,データベースの設計と構築をした.次に,PHPを用いてアプリケーション
と通信を行い,データベースを操作するファイルを作成した.作成したファイルは,機能が多いアプリケーションに加え,
各プラットフォームごとに通信の方法が異なり,全てのメンバと情報共有をすることが困難であった.そこで,サーバ班は
すべてのメンバと通信に関する情報を共有するためにLINEでの会議を行った.この会議では,現在どのような機能があり,
どのプラットフォームとはどのように通信をするのか,現在どこまで実装できているのか,何か問題が発生していないのか
という情報の共有をした.PHP ファイルを完成させた後は,他プラットフォームメンバと相談し,サーバ班は実装された機能と
実際に通信ができているのか通信テストを行い,その後は,デモ用のデータの作成などを行った.



\bunseki{木津 智博(未来大)}
