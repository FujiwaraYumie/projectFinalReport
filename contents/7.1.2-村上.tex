\subsection{村上 惇}
\par
5月には合同合宿に向けたアイディア出しやその合宿に向けたデモアプリケーションの開発を行った.
6月では第一回合同合宿を行い,開発するアプリケーションを決定した.さらに仕様書の作成にも取り掛かった.
7月は要求定義書,要件定義書を書くべく会議を進めると共にサービス仕様書,詳細仕様書の作成にも入り,加えて中間発表の準備にも取り掛かった.中間発表では,発表スライドのリーダとして作成を行った.またここでもデモアプリケーションを作成し,中間発表会に臨んだ.
8月に入ってすぐオープンキャンパスの準備を行った.夏休みである8月,9月ではアプリケーション開発に必要な技術習得やアプリケーションの開発,後半に入ってはビジネスモデルの考案などを行った.
10月はアプリケーションの開発を少し行った.
11月に入っては,アカデミックリンクにポスタとデモアプリケーションを出したり,第二回合同合宿を行った.
12月では,最終発表のためのデモアプリケーションを作成し,最終発表の準備を行った.最終報告書の作成を開始した.
1月は引き続き最終報告書の作成を行うと共に,2月の企業報告の準備に取り掛かる.
5月
\begin{itemize}
\item 合宿に向けたアイディア出し
\item iOS 班リーダに就任
\item 第一回合同合宿用デモ作成(iOS)
\end{itemize}
6月
\begin{itemize}
\item 第一回合同合宿
\item 要求定義書の作成		 
\item 要件定義書の作成
\end{itemize}
7月
\begin{itemize}
\item 要求定義書のための会議に参加
\item 要件定義書のための会議に参加
\item サービス仕様書のための会議に参加
\item 中間発表用のRhyth/Walkのスライド作成
\item 中間発表用のRhyth/Walkのポスターを作成協力
\item 中間発表第1次Rhyth/Walkの台本の作成
\item 中間発表用デモ作成(iOS)
\item 中間発表
\end{itemize}
8月
\begin{itemize}
\item オープンキャンパスの用意
\item 必要な技術を習得
\item アプリケーション開発
\end{itemize}
9月 
\begin{itemize}
\item 必要な技術を習得
\item アプリケーション開発
\end{itemize}
10月
\begin{itemize}
\item 必要な技術を習得
\item アプリケーション開発
\end{itemize}
11月
\begin{itemize}
\item アカデミックリンク
\item 第二回合同合宿
\item 最終発表用の「Rhyth/Walk」のポスターを作成協力
\item 最終発表用の「Rhyth/Walk」のスライド作成協力
\end{itemize}
12月
\begin{itemize}
\item 最終発表用デモ作成(iOS)
\item プロジェクト最終報告
\item 最終報告書の作成
\end{itemize}
1月
\begin{itemize}
\item 最終報告書の作成
\item 最終報告会
\end{itemize}
2月
\begin{itemize}
\item 企業報告会の準備
\item 企業報告会
\end{itemize}

\bunseki{村上 惇(未来大)}
