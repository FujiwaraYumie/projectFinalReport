【スケール解析】
\par
%ここから本文
スケール解析では,高速フーリエ変換(以下,FFT)を用いて楽曲の波形データからスケールを検出するという事を目標に実装作業を行ってきた.スケールとは,各音階の組み合わせの事であり,主にメジャースケールとマイナースケールの二種類に分けられる.一般的に,メジャースケールは明るい印象を,マイナースケールは暗い印象を与えると言われている.スケール解析では,その情報を自動選曲の機能に組み込めないかと,スケールが人にどの様な印象を与えるのかという事の調査も併せて行っていた.現段階で実装できている機能はピアノソロ曲のスケール解析のみであるが,精度としてはかなり高精度の物が出来ている.スケール解析は次の様な手順で行っている.\par
まず,楽曲の波形データにFFTを使用し,周波数成分を割り出す.次に,割り出した周波数を用いてどの音階の周波数が最も強く出ているかを判定する.最後に,判定した音階の組み合わせからスケールを割り出しその楽曲のスケールとする.
