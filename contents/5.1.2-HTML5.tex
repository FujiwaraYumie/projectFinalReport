\subsection{HTML5班}
\par この節では,本プロジェクトのアプリケーションである「Cool Japanimation」を開発するために行ったプロセスについて述べる.

\par 前期のプロセスを述べる
\begin{itemize}
\item HTML5アプリケーション開発のための環境構築
\item アプリケーション開発のための環境構築,技術習得
\par  技術習得の成果として以下の3つの機能を実装したデモアプリケーションを開発した.(未来大)
\par    写真機能
\par    GPS 機能
\par    加速度機能
\par  技術習得の成果としてウェアラブルデバイスからネット上に情報を上げることを利用したアプリケーションを開発した.(長崎大)
\item 画面遷移図の作成
\end{itemize}

\par 後期のプロセスを述べる
\begin{itemize}
\item 開発スケジュールの決定
\item Skypeで毎週進捗確認
\item 作成した画面遷移図を元にアプリケーション開発
\item 画面遷移のみ実装
\item ローカルでの実装
\item サーバとの連携に向けた開発
\item 未実装機能の実装スケジュール(優先度)を決定
\item 未実装機能の実装
\end{itemize}

\bunseki{藤原 由美恵(未来大)}
