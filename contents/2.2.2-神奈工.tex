\subsection{神奈川工科大学}
神奈工はアプリケーションの提案,開発,他大学との潤滑な連携のために以下の課題を決定した.
\par●前期
\begin{enumerate}
\item 開発アプリケーションのアイディア提案
\par 開発するアプリケーションのアイディアをメンバ全員で出し合い,その中から第一回合同合宿にて発表するアプリケーションを選定すること.
\item 開発アプリケーションの決定
\par 第一回合同合宿で発表したアプリケーションを,さらに合宿中で作られた3大学混合の新しいグループでアイディアを練り直すこと.新しいグループで作成したアプリケーション案の中から,投票で開発するアプリケーションを決定すること.
\item アプリケーションの機能選定
\par 開発するアプリケーションに対し各グループを割り当て,3大学合同の会議に参加し,アプリケーションへの理解を深めると共に,アイディアとして出てきた機能の中から実装する機能の絞込みを行うこと.
\item 開発のための技術習得
\par Androidのアプリケーションを作成するために必要な技術を習得すること.
\item 類似アプリケーションの調査
\par 第一回合同合宿にて決定したアプリケーションが提供するサービスに対して,同じようなサービスはどのようなものかを調査すること.
\item 要求定義書の作成
\par 開発するアプリケーションがどのような使用を設けるのかを記したものを,3 校で項目ごとに分割し,割り振られた箇所について合同で作成すること.
\item 要件定義書の作成
\par 開発するアプリケーションのソフトウェア要件をまとめたものを,3 校で項目ごとに分割し,割り振られた箇所について合同で作成すること.
\item サービス仕様書の作成
\par 開発するアプリケーションが,ユーザの視点でどのようにサービスを提供するのかを記したものを,3 校で項目ごとに分割し,割り振られた箇所について合同で作成すること.
\item 前期提出物の作成
\par 前期期間内の活動内容を,各校で割り当てられた項目を分担して,最終的に1つの報告書としてまとめて作成すること.
\end{enumerate}

\begin{enumerate}
\par ●後期
\item デモシナリオの作成
\par 第二回合同合宿で挙げられた各アプリケーションのデモシナリオを作成すること.
\item アプリケーションの必須機能選定
\par 実装する機能の絞込みを行うこと.
\item プログラム動作工程の作成
\par iOS班の作業をスムーズにするためにAndroidのプログラムの動作の工程を示すこと.
\item アプリケーションの機能の実装
\par 選定した機能について未来大,神奈工の開発リーダを中心に各アプリケーションの機能を実装すること.
\item 学内最終発表
\par 行ってきたことをまとめ,発表のスライドを作成すること.
\item 後期提出物の作成 
\par 前期の活動内容に後期の内容を加え,各校で割り当てられた項目を分担して,最終的に1つの報告書としてまとめて作成すること. 

\end{enumerate}

\bunseki{安藤 歩美(神奈工)}
