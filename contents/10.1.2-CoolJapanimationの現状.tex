\subsection{CoolJapanimationの現状}

「Cool Japanimation」の現状は以下のとおりとなっている.

\begin{description}

\item[仕様書の作成状況]\mbox{}\\ 
前期の活動において要求定義書,要件定義書,サービス仕様書,詳細仕様書の作成を行った.
要求・要件定義書については,問題ないが,サービス仕様書については,ビジネスモデルの件や,画面遷移図,UIの件などで,加筆・修正を行わなければならない箇所が存在する.
さらに,詳細仕様書においては,現在,どのボタンを押すと,どの画面に遷移するという,遷移図が機能ごとにしか存在していない仕様書のため,多いに加筆する必要性が有る.

\item[アプリケーションの開発状況]\mbox{}\\ 
「Cool Japanimation」は現在AndroidとHTML5で開発を進めているが,現在進捗が大きく違っている.\par
Androidの進捗状況としては,現在一応動き,本アプリケーションがどんなアプリケーションなのかということを,開発者自らが使ってみせることで表現することができる.
しかし,まだサーバとの連携がうまくいっていないため,決められた動きしか行うことができず,実際にデモ機を使ってもらうことができないという状況である.\par
HTML5の進捗としては,サーバとの連携や,データベースからの検索が可能となっており,現在LODを使って収集できているデータの範囲ならば,実際にデモ機を使ってもらいながら,本アプリケーションがどのようなアプリケーションなのかを知ってもらえるような仕様になっている.
しかし,参加申請許可の機能に関しては,まだログイン情報を活用できるような実装になっていなかったり,ナビ機能や,チャット機能など仕様書と一部違った実装になっている機能があるというのが現状である.

\item[ビジネスモデルの作成状況]\mbox{}\\ 
最初は,11月のキャンパスベンチャーグランプリへの提出を目標に,ビジネスモデルの作成を行った.
\par
これにより,ビジネスモデルを考えていくにしたがっての,根幹を作成することができた.
しかし,まだまだ既存のビジネスモデルにあやかるばかりで,「Cool Japanimation」独自のビジネスモデルを考えられていないのが現状である.
現在は,企業報告会,秋葉原での課外成果発表会に向けて独自のビジネスモデルを考えている最中である.

\item[学外発表会の準備状況]\mbox{}\\ 
秋葉原での課外成果発表会に使用するポスターの作成を検討している.最終報告会で使用したポスターを原案として,改良する予定である.
さらに,秋葉原での課外成果発表会では,アプリケーションの概要のポスターだけではなく,ビジネスモデル専用のポスターを作成する予定である.
\per
加えて,協力してくれた企業への,企業報告会も行う予定である.そこでは,今年一年の成果をスライドによるプレゼン形式で発表したり,デモンストレーションを作成して,実際に「Cool Japanimation」がどのようなアプリケーションになったのかをみてもらう予定である.

\end{description}
\bunseki{小笠原 佑樹(未来大)}
