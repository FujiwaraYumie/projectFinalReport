\subsection{サーバ班の役割}
\par サーバ班はリーダが1名,班員が2名の計3名で構成されている.
以下にサーバ班の活動内容について述べる.
\par <Wikiサーバの構築>
\par
本プロジェクト内における情報や,議事録などのファイルの共有,スケジュール管理はWikiを用いて行った.例年,本プロジェクトではプロジェクト開始時にWikiを動かすためのサーバ構築を行っており,今年もサーバ班ではWikiサーバの構築を行った.本年度は以下の構成でWiki サーバを構築した.また,機密性の確保をするためにWikiへのアクセスをIDとパスワードを設けることにより制限するようにした.
\par ・IPアドレス:210.226.0.74
\par
\par
\par <基本知識の習得>
\par
サーバについての基本的な知識の獲得を行った.主な習得方法としてはOSSセミナの受講による知識獲得であった.具体的にはLAMPを用いてWebサーバを動かすことやサーバのセキュリティ対策を行うこと,Webアプリケーション開発を行うこと,jQueryによる動的なWebサイトの作成を行った.また,サーバ
\par <各アプリケーションサーバの構築>
\par
本プロジェクトで開発することとなった2つのアプリケーション「Cool Japanimation」と「Rhyth/Walk」のアプリケーションサーバの構築を行った.各アプリケーションサーバの詳細については「Cool Japanimation」は5.1.3に「Rhyth/Walk」は5.2.3にそれぞれ記載する.
\bunseki{鍋田 志木(未来大)}
