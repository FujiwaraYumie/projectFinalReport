\subsection{サーバ班の役割}
\par サーバ班はリーダが1名,班員が2名の計3名で構成されている.
以下にサーバ班の活動内容について述べる.
\par <Wikiサーバの構築>
\par
本プロジェクト内における情報や,議事録などのファイルの共有,スケジュール管理はWikiを用いて行った.例年,本プロジェクトではプロジェクト開始時にWikiを動かすためのサーバ構築を行っており,今年もサーバ班ではWikiサーバの構築を行った.本年度は以下の構成でWiki サーバを構築した.
\par
\par ・IPアドレス:210.226.0.74
\par 教員から借りたIPアドレス
\par
\par ・OS:CentOS5
\par WikiサーバのOS
\par 
\par ・仮想:VMware vSphere
\par OSを仮想上で稼働させるためのもの
\par 
\par ・Web:Apache(httpd) 2.1
\par Webサーバを稼働させるWebエンジン
\par 
\par ・PHP:5.4
\par Wikiを動作させるためのもの
\par 
\par ・Wiki: PukiWiki 1.4.7
\par 3大学情報共有のためのWiki
\par
\par
利便性確保のため,グローバルIPアドレスを1つ教員から借りた.無料で登録可能なDNSに登録し,このWikiサーバをインターネットに公開した.ただし機密性確保のため,Wiki 閲覧・編集はパスワードにより保護を行なっている.IPアドレスはひとつしか無いが,ドメイン名でレスポンスを返すサーバを振り分け,複数台のサーバがネットワーク外からアクセスできるように構築した.
\par
また,Wikiサーバをインターネットに公開した以上,その保守・管理も重要となる.セキュリティ確保のため,以下のソフトウェアを導入した.
\par
\par ・ウイルス対策:Clam AntiVirus
\par ウイルスを検知・削除する
\par
\par ・不正侵入対策:DenyHosts 2.6
\par SSHでの不正侵入を検知しIPアドレスを元にアクセスの権限を決定する.
\par
\par ・改ざん防止:Tripwire 2
\par ファイルの変更を検知する.
\par 
\par
日常の管理はSSHによるリモートログインで行なうが,セキュリティを考慮しパスワード認証でのログインを禁止した上で,RSA鍵での認証を行なっている.また,データディレクトリの自動バックアップを行なうシェルスクリプトを作成し,Wiki内のデータ保全を図っている.これらの定期的なログチェックを行ない,不具合発覚時には,バックアップの復元やサーバの調整など適切な対策を行なった.
\par 
\par <基本知識の習得>
\par
サーバについての基本的な知識の獲得を行った.主な習得方法としてはOSSセミナの受講による知識獲得であった.具体的にはLAMPを用いてWebサーバを動かすことやサーバのセキュリティ対策を行うこと,Webアプリケーション開発を行うこと,jQueryによる動的なWebサイトの作成を行った.またサーバ通信についての知識獲得は,ライブラリや
\par <各アプリケーションサーバの構築>
\par
本プロジェクトで開発することとなった2つのアプリケーション「Cool Japanimation」と「Rhyth/Walk」のアプリケーションサーバの構築を行った.各アプリケーションのサーバ班の役割については「Cool Japanimation」は5.1.3に「Rhyth/Walk」は5.2.3にそれぞれ記載する.
\bunseki{鍋田 志木(未来大)}
