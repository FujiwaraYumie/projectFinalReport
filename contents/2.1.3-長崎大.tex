\subsection{長崎大学}
\par
長崎大学は,自由参加授業の一環として創生プロジェクトのスマホアプリ班として今回のミライケータイプロジェクトに参加した.
\par
長崎大学スマホアプリ班の当初の目的は1年を通して何らかの”モノ”を作り,技術を身に付け,モノづくりコンテストと呼ばれるコンテストに出展することが目標であり,目的であった.
\par
しかし創生プロジェクトの一環としてミライケータイプロジェクトに参加してみると,ただ単にアプリを開発するのではなくアプリの提案から,ビジネスモデルや諸々の仕様書の作成,そしてアプリの開発へと至る社会における企画から開発への一連のプロセスを学ぶことができる機会となった.
\par
3大学連携という事で,遠方にいる大学メンバーとの意思疎通や情報の共有など大変なことも多い中での作業を通し,個々人が実際に作業を通して経験を積んでいくことで社会に出て通用するスキルを身に付けることも目標の1つとなった.
\par
また,開発ではHTML5を用いてiOS・Androidに対応するアプリ開発を行い,全メンバーが何らかの機能を担当し責任を持って開発していくことで,それぞれが十分になんらかの技術を身に付けスキルアップすることも目標となる.
\par
以上の目標を達成していく中で,個々人が社会に出ても通用するような技術・知識を身に付けることが長崎大学の最大の目標となった.
\bunseki{立石 拓也(長崎大)}
