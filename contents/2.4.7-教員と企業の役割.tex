\subsection{教員と企業の役割}
\par <教員の役割>
\par 本プロジェクトでは,未来大から5人,神奈工から1人,長崎大から1人,各大学の担当教員がそれぞれ参加している.各大学の教員は,プロジェクト運営において問題が発生した場合において解決する手助け,プロジェクトに関しての知識指導,さらに,成果を発表する機会,発表に使用する機材の提供,スケジュール管理の方法,資料の添削などをサポートしている.

\par <企業の役割>
\par
HP,IDY,KDDI,NTC,NTTdocomo,SoftBank,Y!mobile,サイバー創研,サンドグラスの企業が参加し,協力している.協力企業は,アプリケーションの企画,開発,ビジネスモデル作成における知識・方法・問題解決及び開発に必要な実機デバイスのサポートをしている. 5 月には 2 度講演会を開き,また6月初旬に行われた第一回合同合宿では,各校が考えたアプリケーション案についてのコメントや,アプリケーションのアイディア練り直しの際における指導,そしてプロジェクトを進めていく上で役立つ情報の提供をした.そして, 11 月下旬に行われた第 2 回合同合宿においても,プロジェクトとしてやらなければならないことについてやビジネスモデルについての指導を行ったり,アプリケーション制作以外にも役立つ有益な講演を行った.
\bunseki{三栖 惇(未来大)}
