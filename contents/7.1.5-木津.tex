\subsection{木津 智博}
\par
 本プロジェクトでは,5月から6月にかけて,第一回合同合宿に向けて,今年開発するアプリケーションのアイディア出しを行い,
 第一回合同合宿の場でプレゼンテーションを行うためにアプリケーションアイディアのブラッシュアップやイメージビデオの
 作成を行った.また,合宿リーダを担当し,第一回合同合宿のために必要な資料のリストアップや作成,毎週木曜に合宿リーダ
 会議を行い,合宿では何を行うのか,誰が参加するのか,何が必要なのか,どのようなスケジュールで合宿を行うのかなどについて
 各大学の合宿リーダと話し合うことで決定した.アプリケーション決定後は,7月の中間発表まで,「Cool Japanimation」の
 ポスターリーダを担当し,ポスターに記載する項目とガントチャートの作成を行った.また,「Cool Japanimation」のサービス
 仕様書リーダを担当し,ガントチャートを用いてのスケジュール作成と仕様書項目の決定を行った.後期には11月に行われる
 第二回合同合宿の合宿リーダを担当し,第二回合同合宿のために必要な資料のリストアップや作成,毎週水曜に合宿リーダ会議を行い,
 合宿では何を行うのか,誰が参加するのか,何が必要なのか,どのようなスケジュールで合宿を行うのかなどについて各大学の
 合宿リーダと話し合うことで決定した.合宿後は,12月の最終発表のために発表準備を行った.発表準備では主に「Rhyth/Walk」の
 毎週日曜に行われる会議に参加し,発表には何が必要なのか,どのように発表するのか,発表するにあたり,どのような情報が必要
 なのかなどを話し合った.その後,開発に追われるメンバに代わり,デモで使用するために必要な音楽のパラメータを含む詳細情報,
 音楽の準備,スライドで使用するために必要な情報収集を担当した.
\par
 開発班としてはサーバ班に所属し,前期はサーバとはどのようなものか,どのようにアプリケーションサーバを構築するのか,
 Wikiの設定をどこでどのように変更するのかについて学習した.8月,9月は「Cool Japanimation」のアプリケーションサーバの構築と
 サーバアカウントの作成,アプリケーションサーバに直接接続することができないかネットワークの設定変更について調べた.
 10月,11月には長崎大がMonacaとの通信で問題が発生し,困っていることを一緒に調べたり,何か対策がないか会議を行ったりして
 対処した.また,アプリケーションアイディアを満たすために必要とされるMySQLとPHPを用いたデータベース関連の動作について
 勉強および作成を行った.
\par

5月
\begin{itemize}
\item 合宿リーダに就任
\item 合宿リーダ会議
\item サーバ班に参加
\item 合宿に向けたアイディア出し
\item Wikiの移行
\end{itemize}
6月
\begin{itemize}
\item 第一回合同合宿
\item 「Cool Japanimation」サービス仕様書リーダに就任
\item 「Cool Japanimation」の要求定義を議論
\item 要求定義書の作成
\item 「Cool Japanimation」の要件定義を議論
\item 要件定義書の作成
\item 中間発表「Cool Japanimation」ポスターリーダに就任
\item 中間発表「Cool Japanimation」のスライド作成
\end{itemize}
7月
\begin{itemize}
\item 「Cool Japanimation」のサービス仕様を議論
\item サービス仕様書の作成
\item 中間発表「Cool Japanimation」のスライド作成
\item 中間発表「Cool Japanimation」の台本作成
\item 中間発表
\item 中間報告書の作成
\item 詳細仕様書の作成
\item アプリケーションサーバの構築
\end{itemize}
8月
\begin{itemize}
\item アプリケーションサーバの構築 
\item アプリケーションサーバに直接接続するためのネットワーク設定
\item アプリケーションアイディアを満たすプログラムの作成
\end{itemize}
9月
\begin{itemize}
\item アプリケーションアイディアを満たすプログラムの作成
\item アプリケーションサーバに直接接続するためのネットワーク設定
\item 合宿リーダに就任
\item 合宿リーダ会議
\end{itemize}
10月
\begin{itemize}
\item アプリケーションアイディアを満たすプログラムの作成
\item サーバとアプリケーションの通信テストや調整
\item アプリケーションサーバに直接接続するためのネットワーク設定
\item 合宿リーダ会議
\end{itemize}
11月
\begin{itemize}
\item アプリケーションアイディアを満たすプログラムの作成
\item サーバとアプリケーションの通信テストや調整
\item 合宿リーダ会議
\item 第二回合同合宿
\item 最終発表のための情報収集
\end{itemize}
12月
\begin{itemize}
\item サーバとアプリケーションの通信テストや調整を行う
\item ロゴコンテスト
\item デモに必要な音楽の詳細情報を収集
\item 最終発表
\item 最終報告書の作成
\end{itemize}
1月
\begin{itemize}
\item 最終報告書の作成
\item 企業報告会の準備
\item サービス仕様書の修正
\item 詳細仕様書の修正
\end{itemize}
2月
\begin{itemize}
\item 企業報告会の準備
\item 企業報告会
\end{itemize}

\par
 前期ではアプリケーションのアイディア出しの少人数グループリーダと合宿リーダを兼任し,どのリーダという役割でも,いかに
 相手に考えていることを伝えるか,どのようにしたら相手が考えているように動いてくれるのかについて考えることが多く,
 なかなか相手に自分の意見を伝えることができずに苦労した.また,タイムキープがうまくできず,グループ活動や合宿
 リーダ会議では長時間の活動を行っていたことが多かった.プロジェクトリーダからアドバイスを頂いてからは,予め議題を提示する,
 今回の会議では何を絶対に決めるという目標を掲げることで活動時間の削減をすることができた.
\par
 後期では開発が中心となるため,多くのメンバが裏作業をする余裕があまりないため,情報収集など地道だが必要不可欠な作業に
 積極的に参加した.また,両アプリケーションの機能・仕様書からデータベースを使用することを想定し,サーバ班メンバに伝え,
 個人で学習および準備を始めたのは積極性が出ていることを自分自身で実感することができた.

\bunseki{(木津 智博(未来大)}
