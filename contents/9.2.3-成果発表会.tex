\subsection{成果発表会}
\par
ここでは成果発表会について述べる.
\par
 2014年12月12日に未来大にて前期,後期を通してのプロジェクト学習の成果を発表する成果発表会が行われた.それに向けて11月下旬から今年開発したアプリケーション「Cool Japanimation」と「Rhyth-Walk」の開発を進め,発表で使用するポスターをその補助である発表資料のスライドを作成し始めた。
今回の成果発表会は,未来大だけのメンバで行った.以下に成果発表で用いた各種発表資料の準備内容と役割分担を示す.
\begin{enumerate}
\item Cool Japanimation  (プラットフォーム:HTML5)
\par
 Cool JapanimationはAndroidとiPhoneのスマートフォン端末に実装した.Androidでは実装が間に合わず,発表には実機を見せなかった.プレゼンテーションの際に実装した画面を使おうとしていたが,テキストが小さく遠くから見えないので,スライド用に見やすい画像を作成した.
\par
\item Rhyth-Walk  (プラットフォーム:Android,iOS)
\par
 Rhyth-Walkは機能のアルゴリズムの実装に時間がかかり,歩くテンポに合わせた選曲を行う機能を実装した.プレゼンテーションではアプリのイメージ映像を作成し、分かりやすい発表を目指した.
\item ポスタ(未来大:中司,紺井,三栖)
\par
 成果発表で使用するポスタは,中間報告会同様プロジェクトの概要を示したメインポスタを1枚,アプリケーションケーションそれぞれの説明を示したサブポスタを2枚,合計3枚作成した.成果発表会では,各アプリケーションのビジネスモデルの説明を加えた.作成においては学生間や教員からの助言を参考に時間がある限り作り直し,聴衆から高い評価を得た.以下の図XXにそれぞれのポスタを示す. 
\item 発表用スライド(未来大:小笠原,藤原,鍋田)
\par
 中間報告会では各アプリの実装する機能の説明を中心とした説明を行ったのだが、成果発表会では各アプリの実装した機能の説明を中心とした説明を行ったので、中間報告会と成果発表会との違いが分かりにくくなったプレゼンテーションになってしまった.作成においては2月行われる提供企業先での発表を意識し、作成した.また,中間報告会ではスライドのマージに膨大に時間がかかったので解消すべく,編集人数を少なくし,スライドの拡張子をそろえるようにした. \item 発表者(未来大学生全員)
\par
作成されたスライド,ポスタを用いて聴衆の前で成果発表を行った.発表回数は6回で,メンバ12人がペアを組み,6組で全員が発表を行った.スライドの内容を把握し,アドリブを加えながら発表者それぞれの言葉で行った.中間報告前にプレゼンテーションをメンバで相互レビューし合うなど発表練習を綿密に行った.当日は多くの来場者に本プロジェクトの活動と成果を伝えることができた.
\item 評価シート(未来大:藤原)
\par
評価シートは全体を通しての発表技術と内容,各アプリケーションケーションの魅力とについての評価を10段階評価に自由記入欄を設けた.

\end{enumerate} 
\par
全体評価
\par
ここには評価シートの結果をもとに平均点とコメントをかく.
\bunseki{中司 智朱希(未来大)}


