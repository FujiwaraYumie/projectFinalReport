\begin{jabstract}
今日,日本では携帯電話の普及率は100%を超えている[1]. さらにその中でも,日本のスマートフォン普及率は50%を超えている[2]. 日本国民の全員が携帯電話を持ち,その中でも2人に1人はスマートフォンを利用していることになる.それだけ,スマートフォンは従来の携帯電話と並んで,日本国民には欠かすことのできない重要な携帯デバイスとなっている.
なぜ,それだけスマートフォンが普及しているのか.それは,スマートフォンの利便性にある.スマートフォンは従来の携帯電話と違い,通話・Eメールはもちろんのこと,ゲーム・動画視聴・ブラウジング・スケジュール管理・データ管理・SNS・地図・チャットなど挙げたらきりがないほど様々なことが可能である.
これにより,いままでPCなどを利用しなければできなかったようなことが,外出先でも,手元に収まるほどの小さな携帯デバイスを使うことで可能になったのだ.
そこで本プロジェクトでは,公立はこだて未来大学・神奈川工科大学・長崎大学との3大学合同で,いままでになかった近未来的なアプリケーションの提案,開発を行う.さらに,民間企業とも提携し,端末提供,助言を頂くなどの協力を得ている.
アプリケーションの開発はウォーターフォール型の開発プロセスに乗っ取って,実践的に行う.まずアイディアを提案し,仕様書を書いて設計した後,開発を始め,開発を終了した後は,受入テストを行い,最終的に納品する.また,ビジネスモデルの考案も行う.
このように提案,開発を行っていくことで,実践的な開発プロセスについて学ぶことができる.またこのように学びながら開発を進めることが本プロジェクトの目的となっている.
\bunseki{岩田 一希(未来大)}

参考文献:\\[1mm]
[1]東海管内の携帯電話・PHSの普及状況(平成25年3月末)
\par 
http://www.soumu.go.jp/soutsu/tokai/kohosiryo/25/0531.html
最終アクセス:平成26年7月7日
\\[1mm] 
[2]田中美保 スマホの世帯普及率5割超す 内閣府の消費動向調査,朝日新聞DIGITAL,
\par
http://www.asahi.com/articles/ASG4K51RSG4KULFA01V.html,平成26年4月17日 最終アクセス:平成26年7月7日

\end{jabstract}
