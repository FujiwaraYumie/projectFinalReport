\subsection{Android班の分担}
\par
本プロジェクトでは,「Cool Japanimation」「Rhyth/Walk」の2つのアプリケーションを提案・開発しているが,そのどちらのアプリケーションともに,Androidでの開発をすることになった.
未来大のAndroid開発班発足当時,班内で,両アプリケーションの開発に携わりたいという声があったため,未来大のAndroid班の4人(岩田,澤田,中司,紺井)を,1つのアプリケーションに2人と振り分けるのではなく,4人全員で2つのアプリケーションを開発するという体制を取ることにした.
その結果,機能の多い,「Cool Japanimation」では,一人1~2機能を分担することになった.しかし,「Rhyth/Walk」では,機能をもっと細分化して担当を分ければよかったのだが,澤田にメイン機能すべて,岩田,中司でUIを担当,という分担になり,この時点で,全員が両方のアプリケーションに携わるという前提が守られなくなってしまった.
そしてその後,Android班全体的に,タスクが多くなってしまったり,実際に仕様を話し合ったアプリケーションのほうが,開発しやすかった,さらには,タスクを振るよりも,一人で開発したほうが早いと考えた,などの理由から,結局「Cool Japanimation」に3人「Rhyth/Walk」に1人という分担で開発を進めていくことになった.
神奈工にも「Rhyth/Walk」の開発班が3人いるため,人数比としては,3:4なので問題はないが,当初の目的通り,全員がどちらのアプリケーションに携わる開発ができなかったことはとても残念である.
\bunseki{岩田 一希(未来大)}
