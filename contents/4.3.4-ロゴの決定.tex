\subsection{ロゴの決定}
\par
私たちは,ロゴを決めるのにあたって,未来大で2つロゴの案を出して,ペイントやイラストレーションを使用し作成した.
長崎大は一人一つずつロゴの案を出し,手書きで書かれたものと,ペイントを使いロゴを作成した.
作成機関は12月3日から12月9日までの一週間であった.
提案されたロゴは,Googleドライブにあげ,プロジェクトメンバ全員が閲覧できる状態にし,12月10日(水)に各自が良いと
思ったロゴにに投票して,一番投票数が多いものがCool japanimationのロゴの原案とした.
投票形式は,wikiにCool japanimationのロゴ用のページを用意し,提案されたロゴに番号をつけて,
良いと思ったロゴの番号の投票ボタンをクリックするものとした.
ロゴは,これで最終決定したのではなく,選ばれたロゴをこれから少しずつ改良する方針にすることにした.
選ばれたロゴは,富士山に,習字タッチの文字で書かれたCool japanimationで,日本の日の丸をモチーフにした赤が背景の色として使われていた.
しかし,このデザインのままであると,日本らしさと旅行に行くニュアンスのアイデアが盛り込まれてはいるが,
肝心のアニメらしさを上手く表現出来ていないものであった.選ばれなかったロゴに関しても同じことが言えた.
そこで,私たちは,選ばれた原案を元に改良が必要であると判断した.これからの予定としては,今の原案のロゴをアニメタッチに改良して,
アニメらしさを表現できないか,あるいはキャラクターを考えてつけたしてアニメ感を出そうかと考えている.
\bunseki{金澤 しほり(未来大)}
