\subsection{坂本 豊教}
5月
\begin{itemize}
\item Wikiリーダに就任
\item アプリケーション案の発案
\item サーバ班に参加,PHPの勉強
\item 3大学間で使うWikiを2014年度版に移行
\item 会議やブラッシュアップでのアプリケーション案の絞り込み
\item グループで会議,スライド作成
\end{itemize}
6月
\begin{itemize}
\item 第一回合同合宿
\item サーバ班新リーダー就任
\item OSSセミナーに参加
\item サーバとアプリケーション間で通信をするための技術習得
\par  HTTPメソッドであるPOST,GETについて勉強
\item グループでのRhyth-Walkの検討
\item Rhyth-Walk要求定義書,要件定義書,サービス仕様書の作成
\item Rhyth-Walkサービス仕様書リーダーに就任
\end{itemize}
7月
\begin{itemize}
\item 中間報告書,詳細仕様書の作成
\item 中間発表ポスター,スライドの作成
\item 「Rhyth-Walk」サーバの構築
\item 新美先生に停電時のサーバが行う作業や、サーバの技術習得についての相談
\item 中間報告会
\end{itemize}
8月
\begin{itemize}
\item 「Cool Japanimation」サーバの構築
\item PHP・MySQLの技術習得・検証
\item Wikiのバックアップ
\end{itemize}
9月
\begin{itemize}
\item PHP・MySQLの技術習得・検証
\item クライアントのjava,iOS,HTMLの勉強
\item アプリケーションとサーバ間通信の勉強
\item 最終発表の準備
\end{itemize}
10月
\begin{itemize}
\item PHP・MySQLの技術習得・検証
\item RedMineをアプリケーションサーバに導入,PHPと共存出来ずRedMineアンインストール
\item プロキシのssh転送について調査
\item javaとphpのpost通信に関する調査
\item swiftとphpのpost型送受信に関する調査
\item キャンパスベンチャーグランプリに参加
\item 最終発表の準備
\end{itemize}
11月
\begin{itemize}
\item アカデミックリンクに参加
\item 第二回合同合宿に向け,スライド,作成
\item 第二回合同合宿に参加
\item 最終発表で使用するスライド,台本作成のサポート
\item 最終報告書の準備
\end{itemize}
12月
\begin{itemize}
\item 最終報告書の準備
\item 最終発表
\item アプリケーションのロゴ確定
\item 「Rhyth-Walk」から「Rhyth/Walk」へ変更
\item Wikiのバックアップの予定
\end{itemize}
1月
\begin{itemize}
\item アプリケーションの動作テストをする予定
\item 仕様書の最終見直しの予定
\item 協力企業に提出するDVDコンテンツ集めの予定
\item Wikiのバックアップの予定
\end{itemize}
2月
\begin{itemize}
\item 秋葉原課外発表会
\item 企業報告会
\end{itemize}

\par
活動内容と予定
\par
私は本プロジェクトで初めてサーバを触れることになり,管理や操作の仕方が分からず戸惑った.
はじめに前年度のサーバ担当者にサーバのアカウントとサーバを操作するための鍵を作ってもらった.
仕組みは全然わからなかったが,講義で説明を受けたり,参考書やインターネットでキーワードを検索することによって,
仕組みを勉強した.学内から学外にあるミライケータイプロジェクトのサーバを操作するために
22
443
902
903
の学籍番号に関連したポートを開けてもらった.
はじめはポート自体の意味はわからなかったが,OSSセミナーなどの講義を受けることで,サーバの知識を得ていった.
サーバにアクセスできるようになり,トライ・アンド・エラーを重ねることで,サーバの操作に慣れた.
サーバ班の最初のタスクは,2014用のWikiを作成することだった.
先輩に手伝ってもらいながら,2014用のWikiを作成した.次に行なったことは,WikiのURLを変更したことである.
IPアドレスとドメイン名を結びつけることや,Webサーバについて学ぶことができた.
先輩にサーバ室へ連れて行ってもらい,サーバに関する話を聞いたり,作業を行なった.
わからないことは,先輩や,ミライケータイプロジェクトの担当教員に聞き,問題を解決した.
アプリケーションが決まり,アプリケーションの実装段階になり,何のツールを使い,
何をすべきかわからなくなってしまったので,アドバイザーである新美先生に何度も相談にのってもらった.
相談することで得たキーワードを元にApache,LINUX,MySQLを使ってWebサーバを構築した.
サーバ班のメンバとデータベースを設計し,データベースサーバを構築した.
PHPを使って,データベースを操作できるようにサーバサイドプログラミングを行なった.
日々の活動の中では,メンバ間で交代し議事録を取った.
わかりやすい議事録を取るのを心がけ,細かいところまで記述した.
私は中間発表では,中間発表で使う発表スライドの作成を行なった.
スライドを作成し,先生にレビューをしてもらい,修正を加え,その工程を数回繰り返し,スライドを作成した.
最終発表には,スライド作成メンバとスライドと台本を作成した.
スライドを先輩方や教授からレビューをもらい,何度も直し,作成した.
また,これからアプリケーションのテストを行なう.
サーバに関しては,知識が増えれば増えるほど,面白みが湧くので,これからも様々な問題に直面して,技術を得たいと感じた.
また,本プロジェクトの最終ゴールである,企業報告会がある.
これからは,企業報告会に向け,スライドや,ポスターなどの発表資料や
,アプリケーション自体の改良を行なっていく予定である.
\bunseki{坂本 豊教(未来大)}
