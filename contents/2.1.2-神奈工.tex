\subsection{神奈川工科大学}
\par
神奈工のプロジェクト学習の目標は企画から開発までの工程や,他大学との連携プロジェクトの学習である.
\par
今年も例年の神奈工と同じように,目標のもと,各仕様書の作成から開発まで,ソフトウェア開発全般のことについて学び,理解するということを目標としている.
\par
また今年は,例年に比べて参加人数が非常に少なく,それぞれが少数精鋭の意識を持ち,少ない人数で他大学と同等の作業・開発レベルを維持することというもう一つの目標もある.例年,本プロジェクトに参加する神奈工メンバの人数は9名前後であるのに対し,今年は3名と1/3程度の人数しかいない.そのため,1人あたりの作業量が増え,メンバ1人1人にかかる負担が大きくなってしまっている.
\par
しかし,逆にとらえればその分多くのことについて一人一人が深く関われるということでもある.人数が少ないことをマイナスとして捉えず,プラスとして捉え役割を全うするということを目標としている.
\par
以上企画から開発までの工程や,他大学との連携プロジェクトの学習,少ない人数で他大学と同等の作業・開発レベルを維持することの二つを神奈工の目標としている.
\bunseki{赤木 詠滋(神奈工)}
