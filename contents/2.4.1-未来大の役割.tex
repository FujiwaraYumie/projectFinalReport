\subsection{公立はこだて未来大学の役割}
\par 未来大プロジェクト学習はミライケータイプロジェクトというタイトルのもと,神奈工,長崎大と共同でモバイルアプリケーションの企画と開発,ビジネスモデルの企画を行う.それは,スマートフォン端末を対象とする.さらに,各種センサの活用も視野に入れ,既存のケータイの枠組みにとらわれない新しい発想でアプリケーションを開発する.その一連の活動を通じて,発案から納品までの実質的なソフトウェア開発手法を学ぶことができる.
\par 開発対象であるAndroid,iOS,HTML5の各プラットフォームの特性によって,それぞれの基本性能が共通に動作について勉強することができる.さらに,各アプリケーションにおいて,利用されるサーバ側にクラウドコンピューティングを取り入れたシステム開発を目標とする.3大学合同のプロジェクトを通じて,大人数での活動することや遠隔地での活動による意見の取りまとめや意識共有の難しさ,達成したときの喜び,各作業を終えるたびに人間的に成長し,全員でひとつの目標に向かって努力し続ける精神も養う.
\par 3大学の総合リーダを担い,本プロジェクトのスケジュール管理やタスクマネジメントをし,より完成度の高いアプリケーションを作る事ができるようプロジェクトを推進していく.
\bunseki{中司 智朱希(未来大)}
