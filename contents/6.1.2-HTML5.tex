\subsection{HTML5班}
前期
\\
\begin{enumerate}
\item アプリケーションの機能決定の遅れ
\par
アプリケーションの機能が決定するまでに時間がかかり,どの機能から技術習得を行うかを決めるのが遅れた.各大学の HTML5リーダの話し合いによってまだ優先順位を完全につけることはできないが,必ず使うと考えれる機能について技術習得を行うということで解決した.
\item 長崎大との連携
\par
HTML5での開発は未来大・長崎大で行うので,どう連携していくかを決めなければならなかった.各大学の HTML5リーダの話し合いによって進捗を合わせながら技術習得を行うと決定し,解決した.
\end{enumerate}
\\
後期
\\
\begin{enumerate}
\item マージの際,発生した問題
\par
未来大で担当していた機能についてマージしていたが、うまくいかず時間がかかってしまっていた.長崎大と問題点を共有することで,長崎大でも起こっていた問題だったということがわかった.長崎大のメンバから対策法を教えてもらい解決した.
\item 作業の遅れ
\par
開発が最初に設定したスケジュールよりも遅れてしまっていた.すべてやろうとしたために起こった問題だった.そのため機能に優先度を設定し,ここまでは必ずやらなければならないということを決めた.また,機能を開発する際にどこまでやるのかも優先度を設定することで無理のないスケジュールを立てることで解決した.
\item UIの認識のずれ
\par
ボタンの色や配置,テキストボックスの仕様などが,プラットフォーム間で統一できていなかったので,各機能担当者間で話し合い統一することで解決した.
\item 足りない画面の発生
\par 
機能ごとに担当者を割り振り実装していたため,マージする際にトップ画面や各機能をつなぐ画面があることに気がついた.開発メンバで全体の画面遷移を考え,どの画面が必要か洗い出すことで解決した.
\end{enumerate}
\bunseki{藤原 由美恵(未来大)}
