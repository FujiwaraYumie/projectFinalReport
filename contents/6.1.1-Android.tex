\subsection{Android班}
前期
\\
\begin{enumerate}
\item アプリケーションの機能の認識の遅れ
\par
アプリケーションの機能の決定が長引いてしまったのと,具体的にどのような機能が必要なのかの予測ができなかったため,とりあえず使うであろう,画面遷移,サーバ通信,ボタンアクション,GPSを学ぶことに決定した.
\item Androidアプリケーション開発が皆初心者だったため,事前知識がなかった.
\par
開発環境の構築,Androidプログラミングを学ぶため,参考書を見て各自学習し,知識の共有を行った.
\item 今回2つのアプリケーションをAndroidで並行開発しなければならない
\par
とりあえず,2つのアプリケーションに共通する技術,画面遷移,ボタンアクションを学びつつ,各アプリケーションで必要な技術に関して扱い易いと判断したものから技術習得することにした.
\end{enumerate}
\\
後期
\\
\begin{enumerate}
\item Androidのバージョンの認識のずれ
\par
メンバ間で話し合い,デモに使用する実機のバージョンも考え統一した.
\item eclipseの不具合
\par
メンバ間で不具合の原因,エラーメッセージなどを共有し,対処した.基本的にはeclipseの再インストールにより解決していた.
\item 機能ごとのUIの認識のずれ
\par
ボタンの色や配置,テキストボックスの仕様などが,プラットフォーム間で統一できていなかったので,各機能担当者間で話し合い統一した.
\item 機能マージ時にトップページやマイページなどの機能と機能の繋ぎのページがない
\par
機能のマージ担当者が,マージする際に繋ぎの画面を作成した.
\item サーバ連携実装の遅れ
\par
画面の仕様に固執してしまい,サーバ連携が滞っていたので,サーバとすぐに連携できそうなところから連携していくこととした.
\end{enumerate}
\bunseki{紺井 和人(未来大)}
