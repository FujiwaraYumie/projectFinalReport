\subsection{Cool Japanimationの今後}
\par
今後,アプリケーション「Cool Japanimation」では以下のことを行う.
\begin{description}
\item[各仕様書の修正]\mbox{}\\
今まで作成してきた各4つの仕様書には不備や変更点などがある.そのため,グループで分担し,効率的に仕様書の修正を行う.具体的な変更点としては,アプリケーションの目的や,背景の確定.各機能のUIの画面遷移図の変更,各機能の説明などである.
\item[各プラットフォームごとに同じレイアウトになるようにする]\mbox{}\\
現在開発中のアプリケーションは,技術や時間的な要因,また仕様書の不備により,プラットフォームでレイアウトの統一が図れていない.今後は確定した仕様書に近づけるように開発,修正を行う.
\item[機能の未完成部分の実装]\mbox{}\\
各プラットフォームごとに実装を行なっているが,現在まだ未完成機能がある.今後未完成機能を実装し,全てのプラットフォームでCool Japanimationのアプリケーションの完成を目指す.
\item[マイページの作成とツアー申請許可の実装]\mbox{}\\
Cool Japanimationには,ツアー詳細画面から,ツアーの申請を送ることができ,ツアーの企画者は自身のマイページに届いたツアー申請を,許可・不許可することができる.今後,まだ仕様が定まっていないマイページやツアー申請許可の機能を仕様を確定し,実装していく.
\item[サーバ連携]\mbox{}\\
Cool Japanimationのアプリケーションはサーバ連携が不十分な状態である.そのため,今後長崎大学の学生と協力し,サーバ連携を行ないアプリケーションの完成を目指す.また,現在テストサーバにてサーバ連携を行なっているので,Cool Japanimationのメインサーバに移行する必要がある.
\item[Cool Japanimationアプリケーションのテスト]\mbox{}\\
Cool Japanimationでは,未完成機能の実装が完了した後,テスト項目の作成とテストの実施を行う.テストを実施する目的はアプリケーションのバグを発見するためである.そして,バグを最小限に抑え完成度の高いアプリケーションを開発する.
\item[ソースコードの修正]\mbox{}\\
テストの結果より,発見されたバグや不足している箇所の修正を行う.また,ソースコードの最適化も同時に行いメモリ効率などを考える.これらの工程を行なうことにより,完成度の高いアプリケーションにすることができる.
\item[秋葉原課外発表準備]\mbox{}\\
本発表はポスターセッション形式で発表するため,発表に使用するポスターの作成やデモ機の使用もあるため,デモに向けてアプリケーションの開発を進める.ポスターは最終発表のときに使用したCool Japanimationのポスターを修正する方向で作成する.デモ機に関しては,残りの日数も考慮し,デモに必要な機能に優先順位をつけて,優先順に開発を行う.
\item[企業報告会準備]\mbox{}\\
サポート企業に出向き,企業報告会を行うので,そのための発表資料や発表手順などを考慮し,作成する.サポート企業に今までのプロジェクトの成果を発表できる貴重な機会であるので,どのように説明を行なえばよいか考え,準備を行う.準備の主な内容はスライド作成とアプリケーションの開発である.
\item[成果物の整理]\mbox{}\\
プロジェクトでは,サポートして頂いた企業に成果物DVDを作成する.Cool Japanimationは,ソースコードの行数や,作業時間などを考慮し工数を計算したり,作成したCool Japanimationに関する全ての成果物を提出する準備を行う.
\end{description}
\bunseki{紺井 和人(未来大)}
