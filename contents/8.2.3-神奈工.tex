\subsection{神奈川工科大学}
\par後期ではアプリの開発、および成果発表を主に行った。前期で得られた成果に加えて、開発するにあたって企画段階で必要なものの理解、共同開発の経験、プレゼンテーションの経験の2つの成果を上げることができた。
\par神奈工メンバは音楽解析を行う部分の開発を担当しているが、その開発の最終段階を理解できていなかった。原因として各仕様書の内容不足がある。本来ならウォーターフォールモデルのように仕様を確定させてから開発を行うべきであるが、知識不足により開発しながらの仕様確定が目立った。前もってプログラムの出力するものや構造、挙動などを考える必要があることを実感した。
\par開発は未来大との共同開発によって行われている。コードの管理にはGitHubを用い、神奈工が形式の提案を行った。マージ作業やコメントなどのコード管理が容易になり、開発を進めるうえで有利になっている。
\parプレゼンテーションは第二回合宿、幾徳祭展示、学内最終発表と機会があり、プレゼンに対する抵抗が以前より少なくなった。各プレゼンともに練習することが少なくベストなものではなかった。練習することの重要性も認識できた。
\bunseki{遠藤 崇(神奈工)}
