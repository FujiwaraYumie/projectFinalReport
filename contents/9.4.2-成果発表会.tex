\subsection{成果発表会}
\par
第12回「学生ものづくり・アイデア展in長崎」
「特色ある大学教育支援プログラム」事業の主要イベントとして始められた「学生ものづくりアイデア展」は,3大学工学部の持ち回りで,年1回開催されることになっている.
今年度は本学がホスト校となっており,平成26年12月6日(土)に第12回「学生ものづくり・アイデア展in長崎」が開催された.
各大学からの出展作品数は新潟大学2,富山大学3,長崎大学15であり,それらの作品を応募した全20チームにより魅力のあるプレゼンテーションが行われた.
なお,本展には新潟大学5名,富山大学8名の教職員の皆様にご参加いただくとともに,ホスト校である長崎大学からは数多くの教職員が運営に携わった.
さらに,創生プロジェクトにおいて「課題テーマ」を提案していただき学生の活動を協力にサポートして頂いた企業・自治体の皆様にも,数多くご参加いただいた.
以下にはプログラムに沿って,実況報告をする.
\par
まず午前中に,石松工学部長(長崎大学)の挨拶により本展の開会が宣言された.
続いて行われたショートプレゼンテーションでは,作品を応募した全チームが,約3分という短い時間設定のなかで,自らの作品を説明した.
他大学のチームの準備状況は十分把握していないが,各チームとも素晴らしい発表であったことから,本展のような催しを経験するたびに大きく成長しているのではないかと推測できた.
\par
午後は,ポスタセッションによるコンテストから開始された.
各チームとも,自分たちの特徴を十分に活かしながら,1時間30分という短い時間内で活発にプレゼンテーションをしていた.
その後,「学生ものづくり・アイデア展に何を期待しますか?」というテーマで公開討論会を実施した.
公開討論会終了後,各大学の工学部長からの講評ののち,表彰式を挟んで,黒川福工学部長・教育センター長による閉式の辞で,盛会のうちに終了した.
\par
以下に成果発表で用いた各種発表資料の準備内容と役割分担を示す.
\begin{enumerate}
\item Cool Japanimation
\par
\item 発表用スライド(長崎大:磯野)
\par
スライドを利用したプレゼンテーションが3分しかなかったために,アプリの概要を説明することを主として制作し,細かな内容はその後に行われるポスタセッションで行うことにした.制作したスライドはメンバーと先生とで共有し,意見を出し合い改善した.
\item ポスタ(長崎大:磯野,大鶴,岡本,吉澤)
\par
午後に行われたポスタセッションで使用するポスタを制作した.各機能の部分を担当者が制作した.それを先生に見ていただき意見をいただき,より興味をもってもらえるような見やすいデザインになるようにした.
\item スライド発表(長崎大:磯野)
\par
3分という限られた時間で発表を行った.内容を詰め込みすぎた分,少し早口になってしまった部分があった.
\item ポスタセッション(長崎大全員)
\par
1時間30分の間,プレゼンテーションを行った.
ここではポスタを貼って説明するのはもちろん,スライドを使用するためのモニタ1台,Android端末とApple端末の2台を用いて説明を行った.
先ほどのプレゼンテーションでは説明できなかったアプリの機能についての細かな説明や,製作技術に関して説明を行った.
主に,教職員の皆様からは言語に対する問題点や意見を多くいただいた.自分たちでは見つけることのできなかった課題や,よりよいアプリにするための意見などをいただくことができた.
\end{enumerate} 
\par
\bunseki{吉澤 健太(長崎大)}

