\subsection{長崎大学}
\begin{enumerate}
\par 長崎大はアプリケーションの提案,開発,未来大と神菜工との潤滑な連携のために以下の課題を設定した
\item ●前期
\item 1.開発アプリケーションのアイディア提案
\par開発するアプリケーションのアイディアを,メンバ間で持ち寄り,その中から第一回合同合宿にて発表するアプリケーションを選定すること.
\item 2.開発アプリケーションの決定
\par 第一回合同合宿で発表したアプリケーションを,さらに合宿中で作られた2校合同の新しいグループでアイディアを練り直すこと.新しいグループで作成したアプリケーション案の中から,投票で開発するアプリケーションを決定すること.
\item 3.開発アプリケーション名の決定
\par 第一回合同合宿で決定したアプリケーションに,アプリケーションの内容を踏まえた上で,考察し名前の案を提出すること. 
\item 4.アプリケーションの機能選定
\par 開発するアプリケーションに対し各グループを割り当て,3校合同の会議に参加し,アプリケーションに対し理解を深めると共に,アイディアとして出てきた機能の中から実装する機能の絞込みを行なうこと.
\item 5.アプリケーション開発技術の習得
\par 開発するアプリケーションに必要な技術を習得すること.MonacaにおけるHTML5を利用したアプリケーションを作成する技術と,サーバを構築する技術を習得する.
\item 6.ビジネスモデル作成技術の習得
\par 開発するアプリケーションを用いたビジネスモデルを作成するための知識・技術習得を行なうこと.
\item 7.類似サービスの調査
\par 第一回合同合宿で決定したアプリケーションが提供するサービスに対して,同じようなジャンルのサービスはどのようなものかを調査すること.
\item 8.要求定義書の作成
\par 開発するアプリケーションがどのような仕様を設けるのかを記したものを,2校で項目ごとに分割し,割り振られた箇所について合同で作成すること.
\item 9.要件定義書の作成
\par 開発するアプリケーションのソフトウェア要件をまとめたものを,2校で項目ごとに分割し,割り振られた箇所について合同で作成すること.
\item 10.サービス仕様書の作成
\par 開発するアプリケーションが,ユーザーの視点でどのようにサービスを提供するのかを記したものを,2校で項目ごとに分割し,割り振られた箇所について合同で作成すること.
\item 11.前期提出物の作成
\par 前期期間内の活動内容を,各校で割り当てられた項目を分担して,最終的に1つの報告書としてまとめて作成すること.
\item ●後期
\item 1.デモシナリオの作成
\par 第二回合同合宿で挙げられた各アプリケーションのキラーシーンを基本とし,デモシナリオを作成すること.
\item 2.アプリケーションの必須機能選定
\par 第二回合同合宿で挙げられた各アプリケーションのキラーシーンに対し,実装する機能の絞込みを行なう.前期で行なった全体的な機能の選定に対し,キラーシーンに必要な最低限の機能に絞ることを目的とする.
\item 3.サーバ処理の実装
\par 「Cool Japanimation」のサーバ班はサーバ処理を実装すること.
\item 4.アプリケーションの機能の実装
\par 選定した機能について各機能の開発者を中心に各アプリケーションの機能を実装すること.
\item 5.ビジネスモデルの構築
\par 各校で協力し,各アプリケーションのビジネスモデルを構築すること.
\item 6.受け入れテストの実装
\par 仕様書を参考に,意図したとおりのアプリケーションが完成したかをテストすること.
\item 7.デバッグの実装
\par テスト項目を満たせなかった機能を,項目の内容を満たすように改善し変更すること.
\item 8.学内中間報告
\par これまでに作成したきた資料およびデモ機を利用して、活動報告を行うこと.
\item 9.学内最終報告
\par これまでに作成してきた資料およびデモ機まとめて,報告すること. 
\item 10.後期提出物の作成
\par 前期の活動内容に後期の内容を加え,各校で割り当てられた項目を分担して,最終的に1つの報告書としてまとめて作成すること.
\end{enumerate}
\bunseki{磯野 祐太(長崎大)}
