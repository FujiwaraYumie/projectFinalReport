\subsection{ロゴの決定}
\par Rhyth/Walkのロゴは開発メンバが中心となってコンテストを行う形式で決定した.コンテストは4回にわたって行われ,1次案は8名14案から投票を行った.それ以降は1次案での獲得票数と重み付けをした集計結果を基にブラッシュアップを行った.2次案は3案をベースに6名19案,3次案は1案をベースに1名35案,4次案は1名12案あった.4次案での集計結果を基に最終的なロゴを決定した.
\par ロゴの作成には手書き,ペイントソフト,ベクタグラフィックスソフトが使われた.製作期間はいずれもは2日程度,投票期間は1日程度である.製作にあたっては参考資料としてコーポレートアイデンティティの定義を確認した.
\par 最終的に決まったロゴはイヤホンで音楽を聴く人と本アプリで特徴的なシークバー,歩くことをイメージした足跡,そしてゆるやかな曲線の五線譜に「Rhyth/Walk」の文字で構成される.音楽を聴く人の背景には歩く道をイメージした線がある.これらの要素でアプリケーションのコンセプトである歩く人に音楽を楽しませることを示している.
\bunseki{遠藤 崇(神奈工)}
