\subsection{第一回合同合宿}
\par
概要
\par 
第一回合同合宿は,未来大,神奈工,長崎大の3大学のプロジェクトメンバが親睦を深めると共に,本プロジェクトで開発するアプリケーションのアイディアの絞り込み及び決定を行うため,6月7日,6月8日の2日間に渡り,神奈川工科大学にて行った.本章では第一回合同合宿の日程,成果,反省についてまとめる.
\par
日程
\par
【会議第1部】
\begin{itemize}
\item 合宿までに各大学が行ってきた技術習得のデモンストレーション
\item 合宿前に各大学で考えたアイディアのプレゼンテーション
\item プレゼンテーションに対する質疑応答
\end{itemize}
【会議第2部】
\begin{itemize}
\item 投票によるアプリケーションのアイディアの絞り込み
\item 各大学混成の新グループを編成し,提案されたアイディアの質の向上および内容の充実を目的に話し合い
\item プレゼンテーションの準備
\end{itemize}
【会議第3部】
\begin{itemize}
\item 混成グループによるアイディアのプレゼンテーション
\item 評価シートを用いて,アイディアの評価
\item 開発するアプリケーションの決定
\end{itemize}
<成果>
\par  第一回合同合宿を通じて得た成果は大きく2つある.1つはメンバ間の交流である.メンバ間の交流についてであるが,今後1年間プロジェクト活動を行っていく上で,地理的条件からSkypeを用いた3大学合同会議が必須である.しかし,インタネット上でのやりとりだけでは,互いの顔もあまり分からない上に,相手の性格も良く分からないといった事態を招き,各校の間に壁が生じていた.今回の第一回合同合宿を通じ,お互いの意見を言ったり,親睦会を行うことにより,メンバ同士の交流が深まり,各大学の壁を取り除くことが出来たのではないかと感じている.もう1つは新たなアイディアの提案だ.学んでいる環境も内容も全く違う大学のメンバ同士での話し合いにより,より多方向からの視点での意見が見られた.それらの意見を上手く反映することにより,アイディア原案とは違う新しく斬新なアイディアを生むことができた.
\par
<反省>
\par
合宿における反省点を反省文としてプロジェクトメンバ全員が書いた.本項目では良かった点・学んだ点と悪かった点・改善したい点の2つに分け,プロジェクトメンバの反省文をまとめ,以下に記述する.
\par
良かった点・学んだ点
\begin{itemize}
\item ほぼ初対面の他大学の学生と自然にコミュニケーションを取ることができ,遠慮なく意見を交わすことができた.
\item 積極的にメモを取る姿勢ができていた.
\item プレゼンの話し方やスライドの流れが良かった.
\item スライドの作り方によって人への伝わり方・評価が変わることを学んだ.
\item 計画をたてるときに逆算することが重要だと学んだ.
\item 短い時間の中で工夫されたプレゼンを仕上げたこと.
\end{itemize}
\par
悪かった点・改善したい点
\begin{itemize}
\item マイルストーンを守ることが出来なかった.
\item 効率よくタイムテーブルを決めて作業をすれば徹夜での作業をせずに済んだのではないか.
\item 合宿リーダに任せっきりではなく,個人でもスケジュールを把握しておく必要がある.
\item 決められたスケジュール通りに行事を行うことが出来なかった.
\item PDCAサイクルを回す時間が長かった.
\item 企業の方にほとんど質問をすることが出来なかった.
\item リーダに頼る面が多々あった.
\end{itemize}
\par
まとめ
\par
今回の第一回合同合宿ではCool Japanimation,Rhyth/Walkを本プロジェクトで開発していくことに決定した.第一回合同合宿には未来大,神奈工,長崎大の三大学の学生・教員に加え,サポートして下さる企業の方々にも参加して頂いた.今回の合宿を通して,プロジェクトメンバ同士の意識の共有や親睦を深めることが出来たことは,とても有意義なものになった.第一回合同合宿を通して,グループワークの進め方やコミュニケーションの取り方,質問事項の考え方などを学ぶことが出来た.その一方で,反省点として自分が次に何をするのかスケジュールを把握していなかった,決められた時間内に作業を終わらせるスケジュール管理がうまく出来なかったという点を挙げている人が多くみられた.合宿を通して得たもの,良かった点・学べた点はそのまま引き継ぎ,反省点・改善点は,どうしたら改善できるのかを意識しつつ,今後のプロジェクト活動に活かしていきたい.
\bunseki{木津 智博(未来大)}
