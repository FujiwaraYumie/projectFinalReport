\subsection{ミライケータイプロジェクトの今後}

\begin{description}

\item[各仕様書の修正]\mbox{}\\ 
現在の仕様書は,前期時点で凍結した段階のままとなっており,後期になって開発をはじめてから
新しく検討した内容などが含まれていない.
したがって,仕様書の不備や不足部分を洗い出し,仕様書の修正を行う必要がある.

\item[機能の未完成部分の実装]\mbox{}\\ 
両アプリケーション共に,まだ完成していない実装機能が存在している.したがって,企業報告会にむけて開発を継続し,
アプリケーションの完成を目指す必要がある.

\item[ソースコードの修正]\mbox{}\\ 
現在のソースコードは,実装を第一と考えて開発を行ってきたために,最適化されているとは言いがたい.
したがって,アプリケーションのテストを行って動作確認を行いつつ,リファクタリングによってソースコードそのもや
メモリ効率の最適化を行わなければならない.

\item[秋葉原課外発表の準備]\mbox{}\\ 
ポスターセッション形式で,プロジェクト学習の成果を発表する.
発表には未来大学の最終報告会に使用したポスターを改良して使用する予定である.

\item[企業報告会の準備]\mbox{}\\ 
2月初頭に協力企業へこの1年間のプロジェクト活動の成果報告を行うために,発表資料の作成や練習を行う.
同時に,開發したアプリケーションを用いたデモを行うので,開発を進めるとともに,デモの発表の方法を検討する.

\item[成果物の整理]\mbox{}\\ 
仕様書や報告書などのドキュメントや発表の際に使用したポスターやスライドなどの資料,
その他アプリケーションのソースコードなどのこのプロジェクト学習で作成してきたすべての成果物を整理する.
整理した成果物は,企業報告会の際に納品するDVDに書き込んで企業報告会の場で協力企業に納品する.
\end{description}

\bunseki{小笠原 佑樹(未来大)}
