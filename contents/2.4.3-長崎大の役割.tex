\subsection{長崎大学の役割}
\par
長崎大学は今回情報システムコース4年生3人、機械コース3人の計6人が今回のプロジェクトに参加している.
構造化されたデータ同士をリンクさせることでコンピュータが利用可能な「データのウェブ」の構築を目的とし,
セマンティック・ウェブ分野で検討されてきた知識の構造化手法を既存の大規模データに適用し,
コンピュータで処理可能なデータを普及させるための一連の技術ならびに方法論に則ったLoDを使ったアプリケーションを開発していく中で,
他大学との連携をとりソフトウェア開発の仕方を学ぶ.
今回のプロジェクトでは,Cool Japanimation班でHtml班とサーバ班に別れて未来大学との開発を行ってきた.
全体の共同作業としては,第一回合宿でのでも発表,要求定義書・要件定義書・サービス仕様書・詳細仕様書作成などを行い,
第二回合宿では,開発していく過程で生じた仕様書とのズレを元に,互いの大学の意識の共有と今後の修正課題を明確にした.

\bunseki{清水 彰人(長崎大)}
