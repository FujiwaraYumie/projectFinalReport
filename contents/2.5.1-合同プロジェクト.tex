\subsection{合同プロジェクト}
\par
本プロジェクトにおける各大学の役割分担としては,アプリケーションの企画,設計,開発を
未来大,神奈工,長崎大が分担して担当する.
情報の共有はSkypeでの合同会議やメーリングリスト,Wikiなどで行い,成果物はWikiに添付する形で共有した.
\par
以下は,未来大,神奈工,長崎大が行った合同ミーティングの活動スケジュールである.
\par
・前期
\begin{itemize}
\item 4月30日 :神奈工と初顔合わせ,自己紹介,今後のスケジュール確認
\item 5月14日 :長崎大と初顔合わせ,第一回合同合宿に向けての話し合いと準備
\item 5月21日 :専修大学 渥美幸雄様の講演
\item 5月28日 :合宿までのスケジュールを確認,第一回合同合宿の準備
\item 6月11日 :アイデア絞込,デモ開発に関して各大学の進捗を確認
\item 6月18日 :各大学の仕様書の進捗・〆切の確認
\item 6月25日 :各大学の仕様書の進捗・〆切の確認,中間報告書について提起
\item 7月 2日 :各大学の仕様書の進捗・〆切の確認,中間報告書添削のお願いを提起
\item 7月 9日 :各大学の仕様書の進捗・〆切の確認,夏休みの活動について提起
\item 7月23日 :夏休みの活動について提起,決定.大学間の開発体制の決定
\end{itemize}

前期は,大学同士の顔合わせをするとともに,スケジュールの確認や連携体制の模索を中心に議題を設定して活動を行っていた.
また,仕様書の作成は3大学すべてのメンバが混合で行ったので,各メンバの進捗の確認や〆切の確認などを定期的に行っていた.

\par
・後期
\begin{itemize}
\item 10月10日 :夏休みの間の進捗確認,プラットフォームごとの開発状況の報告と〆切の確認
\item 10月22日 :プラットフォームごとの開発状況の報告,第二回合同合宿の提起
\item 10月29日 :プラットフォームごとの開発状況の報告,第二回合同合宿に向けての話し合い
\item 11月 5日 :第二回合同合宿に向けての話し合い
\item 11月12日 :第二回合同合宿に向けての話し合い,プラットフォームごとの開発状況の報告
\item 11月19日 :第二回合同合宿に向けての話し合い,プラットフォームごとの開発状況の報告
\item 11月26日 :企業報告会の提起,最終報告書作成の提起,冬休み期間の確認,第二回合同合宿の振り返り
\item 12月 3日 :最終報告書作成の提起,受入テストの提起
\item 12月10日 :最終報告書作成の話し合い,受入テストの話し合い
\end{itemize}

後期は,実際にアプリケーションの開発を行う行程をメインに活動していたため,合同会議でも
それぞれのプラットフォームでの開発の進捗状況を確認する話し合いを多く行っていた.
また,開発を行う上でアプリケーションの詳細について話し合う必要が出てきたために,
後期の合同会議は,はじめに全体で確認する必要のある内容を話し合い,その後Rhyth/Walkについて未来大と神奈工が,
Cool Japanimationについて未来大と長崎大が平行してSkype会議をつなぎ,効率よく話し合いを行うことが出来るようにした.
\bunseki{小笠原 佑樹(未来大)}
