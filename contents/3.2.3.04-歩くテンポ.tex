【歩くテンポ】
\par
本システムがシチュエーションとして取得している"歩くテンポ"の取得方法について述べる.本システムでは,単位時間あたりの歩数を取得し,その歩数を1分間あたりの歩数に換算したものを歩くテンポとする.
\par
歩数は,3軸の加速度センサを用いて取得する.具体的には加速度センサーの値が変更されると,前回との差分を計算し,そのベクトル量の変化をもとめる.ベクトル量はピタゴラスの定義を用いて求め,閾値を超えるベクトル量がある場合に歩数としてカウントする.
\par
歩くテンポは上記の方法で取得した歩数を,計測した時間で除算し,その商を1分間に換算することで得る.式は次のとおりである.歩くテンポ = (単位時間あたりの歩数[歩] / 計測時間[ms]) * 60000 [ms]
\bunseki{澤田 隼(未来大)}
