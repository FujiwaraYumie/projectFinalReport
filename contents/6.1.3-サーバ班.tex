\subsection{サーバ班}
\begin{enumerate}

\item サーバ構築の問題
\par
本アプリケーションは未来大と長崎大の両サーバ班が担当するが,どちらの大学がどこにサーバを構築するのか,という
点において両大学で意思疎通の齟齬が生じた.大学の研究室側の問題で,アプリケーションサーバの構築をすることが
できないという理由が判明し,未来大でアプリケーションサーバの構築をおこなうことで問題を解決した.

\item サーバへの直接接続の問題
\par
Teratermなどのターミナルを用いて,直接アプリケーションサーバに接続することができないかと情報収集・試行錯誤をおこない,
教員にも相談したが,設定ファイルが見つからず,直接アプリケーションサーバへの接続ができなかった.対処法として,
未来大でアカウント作成およびアプリケーションサーバへWikiサーバからSSH接続で経由することでアプリケーションサーバに
接続ができるようにし,問題を解決した.

\item HTML5との通信問題
\par
Cool JapanimationではPOST通信を行うことを当初想定していたが,HTML5の開発環境であるMonacaでは,post通信ができないことが
判明した.LINEでの会議で原因が判明後,他の通信方法としてAjax通信を用いることで対処し,問題を解決した.

\item データベースとの外部通信問題
\par
同一ネットワーク内だと,PHPを使ったMySQLへのデータ格納およびajax通信により,サーバ側からMonacaへデータを受け渡して表示する
ことが可能だが,未来大のサーバを使い外部からのアクセスで失敗するという問題が発生した.教員に相談し,頂いたアドバイスを
もとに,長崎大で再挑戦したところ通信に成功し,問題を解決した.原因は現在でも不明だが,恐らく接続先の指定で
問題があったと想定される.

\item アニメ検索のHTMLファイルの効率化問題
\par
アニメ検索で表示する各アニメの詳細情報が記載されたページを非効率的だが,各アニメ毎にページを用意していた.
アニメのページへ飛ぶときに,URLにパラメータを持たせ,LODEndpointにSPARQLクエリを飛ばすことでhtmlファイルを2つまで
削減することに成功し,問題を解決した.

\item リンク先へ飛ばす画像表示問題
\par
アニメの詳細情報が記載されたページへ飛ばすリンクを画像にできず,文字を記載していた.対処法として,
画像URL自体をEndpointにMySQLのように格納し,アニメ情報を引っ張るときに同時に持ってくることでアニメごとの
ロゴ画像を表示することに成功し,問題を解決した.

\end{enumerate}
\bunseki{木津 智博(未来大)}
