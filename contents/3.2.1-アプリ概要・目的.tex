\subsection{アプリケーション概要・目的}
\par
ここでは,開発アプリケーションRhyth/Walkの目的・概要について説明する.
本プロジェクトで開発するアプリケーションの一つであるRhyth/Walkは,音楽を聴取する際に選曲する煩わしさを無くし,Rhyth/Walk使用者(以下:ユーザ)がスムーズに音楽を聴取することができるように手助けをするということ,
またその際,状況にマッチングした音楽をユーザに提供することにより,ユーザがより気分良く音楽を聴取することができる様に演出し,ユーザが音楽を更に楽しめるように手助けすることを目的としている.
\par
本アプリケーションRhyth/Walkでは,歩くテンポ,時間,季節,場所,天気からユーザのシチュエーションを判断し,
ユーザのシチュエーションに対して適切な音楽データを自動で選曲する.
選曲された音楽をユーザに提供する事により,ユーザが現在の状況をより楽しめ,かつ普通に音楽を聴取する以上の楽しさをユーザに提供する..
また,自動選曲機能により,ユーザが音楽を選択する煩わしさを無くす.
\par
さらに,「ユーザのシチュエーションのどの要素を重視してほしいか」ということをユーザに指定してもらう「優先度設定機能」により,よりユーザが求める音楽の自動提供,および演出をすることができる.
\bunseki{赤木 詠滋(神奈工)}
