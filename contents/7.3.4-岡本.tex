\subsection{岡本 優介}
5月
\begin{itemize}
\item プロジェクトに参加
\item 合宿に向けたアイディア出し
アプリの概要やコンセプトについて考えた.
\item 外国人に対してのアンケート調査
外国に住んでいる友達とその知り合いにアプリのアンケート調査を行った.
\end{itemize}
6月
\begin{itemize}
\item Cool JapanimationのHTML班に参加
\item Cool Japanimationの要求定義を議論
要求定義の内容を話し合った.マッチング成功時の連絡先通知機能を担当した.
\item 要求定義書の作成
決められた担当箇所の定義書を作成した.
\item Cool Japanimationの要件定義を議論
要件定義の内容を話し合った.マッチング成功時の連絡先通知機能を担当した.
\item 要件定義書の作成
決められた担当箇所の定義書を作成した.
\end{itemize}
7月
\begin{itemize}
\item サービス仕様書の作成
お気に入り機能を担当した.画面遷移図を作成した.
\item 中間報告書の作成
類似アプリケーションの調査を行い,担当箇所の報告書を作成した.
\item Cool Japanimationのアカウント機能を担当
monacaをダウンロードし開発環境を整え,javascriptについて学習を始めた.
\end{itemize}
8月
\begin{itemize}
\item HTML5によるCool Japanimaitionの開発
アカウント機能の開発を進めた.
\end{itemize}
9月
\begin{itemize}
\item HTML5によるCool Japanimaitionの開発
アカウント機能の開発を進めた.
\end{itemize}
10月
\begin{itemize}
\item HTML5によるCool Japanimaitionの開発
アカウント機能の開発を進めた.
\end{itemize}
11月
\begin{itemize}
\item創成プロジェクトの中間発表
\item 	第二回合宿の参加(Skypeにて)
Skypeで第二回合宿に参加した.
\item	創成プロジェクトの発表の準備
スライド,ポスター,デモ,アプリケーションの紹介の練習を行った.
\end{itemize}
12月
\begin{itemize}
\item 創成プロジェクト発表
\item Cool Japanimationの最終報告書の議論
Cool Japanimationの類似アプリケーション調査を担当することになった.
\item 最終報告書の作成
担当箇所を作成した.
\item最終報告書の第一次案の修正・添削
学生内でレビューを行い,修正を加える予定.
\end{itemize}
1月
\begin{itemize}
\item 最終報告書第一次案の修正・添削
先生方から意見をもらい,修正を加え第二次案を作成する予定.
\end{itemize}
2月
\begin{itemize}
\item 企業報告会
\end{itemize}
\per
活動内容
 私はHTML5班の一員としてプロジェクトに参加した.アカウント機能を担当し,HTML5について一から学習を始めた.しかし情報系の人との理解
 の差が大きく,多くの機能開発を任せることになってしまった.外国人に対するアンケートを行った所,やはり日本人が何気なく思っているこ
 とも外国人から見ると魅力的に感じることがあるなど,いろんな視点から物事を捉えることの大切さに気づいた.始めはしっかりアプリの構想
 を考えていたつもりだったが,開発を進めていくうちに変更したほうが良い所や,実装が難しい所などがでできて,見積もりをすることや開発の
 難しさを学んだ.第二回合宿会議ではSkypeによる参加で,離れた場所から連絡を取り合うことの難しさを知った.機械系なので今回のプロジェ
 クトにおける技術的な所は今後活かせる機会があるかはわからないが,スケジュールの大切さや進捗状況をこまめに確認することでスケジュー
 ルの遅れなどに対応し,柔軟に対応していくことなどをこれからの研究などに活かしていきたい.また,企業報告会に向けて担当している機能の
 改良を行う予定である.
 
\bunseki{岡本 優介(長崎大)}
