\subsection{ビジネスモデル説明}
\par
この節では、プロジェクトで企画・開発を行ったアプリケーション「Rhyth-Walk」に対して、考案したビジネスモデルについて説明する。
\par
Rhyth/Walkでは、「アプリケーション内でのCM放送」と「楽曲のアフィリエイト」,「広告による収入」,「音楽レーベルとの提携による楽曲販売」といった4つ収益モデルを作成,考案した.\par
有料版アプリケーションは,無料版のものに追加された機能が使えるようになるものである.追加される機能には選曲に使用されるシチュエーションの追加や,他のユーザが本アプリケーションを用いてどの場所で何の曲を聴いているかが分かる機能を考えている.
\par
\item アプリ内でのCM放送
Rhyth/Walkは音楽再生アプリケーションであることから,画面を見ずとも,広告を見せる方法として,地元店舗と提携することによってラジオのようにアプリケーション内でCM放送をすることによって広告としての利益を得ることができる.
\par
\item 楽曲のアフィリエイト
まずアフィリエイトとは,自分のサイトやブログなどで広告主の商品やサービスを紹介することで,成果があがった場合に報酬を受け取ることが出来る仕組みのこと言う.そして楽曲のアフィリエイトとは,itunesなどの音楽配信サイトの楽曲を本アプリケーションで紹介することによって,楽曲が購入された場合に利益を得ることができる。
\bunseki{三栖 惇(未来大)}
