\subsection{ビジネスモデル説明}
\parここでは,Rhyth/Walkのビジネスモデルについて述べる.ビジネスモデルは未来大におけるキャンパスベンチャーグランプリの成果物にてサービス仕様書にあにあったビジネスモデルを具体化させた。
\par本アプリケーションの事業拡大には3つのフェーズを考えている。まずはユーザの獲得である。ここではまだ本アプリケーションをインストールしていないユーザの獲得に努める。次に提携企業の獲得である。提携企業はアプリケーション内に表示する広告、またはCMソングを提供する企業を想定している。最後は課金ユーザの獲得である。無課金ではCMソングはサビのワンフレーズのみ再生であるが、曲を購入することでフル再生できるようにする予定である。
\par収益は3つの方法を考えている。1つ目は音楽によるアフィリエイトである。これは音楽配信サイトと連携し、ユーザがアプリケーション内で音楽を購入した場合に、購入額の数%を配当として受け取れるものである。促進のためにマッチした曲を視聴できるようにする予定である。2つ目は広告収入である。広告にはクリック報酬型を用いる。これをアプリケーションの空いているスペースにバナー広告として設置する。3つ目は地元店舗との提携である。地元店舗はユーザが住んでいる地域の店舗を想定している。ユーザが提携済みの店舗の近くにいると、店舗のCMが流れるものものである。このサービスはアプリケーションの配信開始から2年目で行う予定である。
\par支出は人件費、初期開発費、宣伝広告費、サーバ維持費、テナント賃貸料の5つを考えている。
\par収支予測では2年目に年間収支がプラスへ、4年目には総合収支がプラスへ、5年目には総合収支が3000万円となる見込みである。
\bunseki{遠藤 崇(神奈工)}
