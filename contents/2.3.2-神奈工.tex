\subsection{神奈川工科大学}
\begin{enumerate}
\item 開発アプリケーションのアイディア提案
\par 開発するアプリケーションのアイディアをメンバ全員で出し合い,その中から第一回合宿の神奈工アプリケーション案として1つ決定した.
\item 開発アプリケーションの決定
\par 第一回合宿で発表したアプリケーションを,各校のメンバを含め新たに組んだ混合班で意見を出し合い,発表した.その後投票で開発するアプリケーションを決定した.
\item アプリケーションの機能選定
\par 班の会議でアイディアとして出てきた機能から,実装する機能の絞込みを行い,決定した.班の構成については後述する2.4.2組織形態に記述する.
\item 開発のための技術習得
\par サーバと通信するアプリケーションを制作するなど,開発するアプリケーションに必要な技術を学んだ.
\item 類似アプリケーションの調査
\par 第一回合同合宿で決定したアプリケーションが提供するサービスに対して,同じような機能のサービスはどのようなものかを調査した.
\item 要求定義書の作成
\par 開発アプリケーションにどのような仕様を設けるかを記したものを,3大学で項目ごとに分担して合同で作成した.
\item 要件定義書の作成
\par アプリケーションのソフトウェア要件をまとめたものを,3大学で項目ごとに分担して合同で作成した.
\item サービス仕様書の作成
\par アプリケーションがユーザ視点でどのようなサービスを提供するかを記したものを,3大学で項目ごとに分担して合同で作成した.
\item 前期提出物の作成
\par 前期期間内の活動内容を,3大学で項目ごとに分担して,合同で中間報告書を作成した.
\end{enumerate}

\begin{enumerate}
\par ●後期
\item デモシナリオの作成
\par 第二回合同合宿で挙げられた各アプリケーションのデモシナリオを作成した.
\item アプリケーションの必須機能選定
\par アプリケーションのリーダを中心に,実装する機能の絞込みを行った.
\item プログラム動作工程の作成
\par iOS班の作業をスムーズにするためにAndroidのプログラムの動作の工程を示した.
\item アプリケーションの機能の実装
\par 選定した機能について未来大,神奈工の開発リーダを中心に各アプリケーションの機能を実装した.各班の構成については後述する2.4.2 組織形態に記載する.
\item 学内最終発表
\par 行ってきたことをまとめ,発表のスライドを作成した.
\item 後期提出物の作成
\par 前期の活動内容に後期の内容を加え,各校で割り当てられた項目を分担して,最終的に1つの報告書としてまとめて作成した.
\end{enumerate}
\bunseki{安藤 歩美(神奈工)}
