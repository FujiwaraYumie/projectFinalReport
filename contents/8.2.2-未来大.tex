\subsection{未来大}
\par
8.2.2 公立はこだて未来大学
\par
ここでは未来大での成果を述べる.
\par
未来大の成果は,大きく分けて5 つある.
1つ目は,アプリケーションの一貫した開発工程とその手法について学んだことである.私たち
は,要求定義,要件定義,実装,受け入れテストなど企業で行なわれるウォーターフォール・モデ
ルという実際の開発手法を用いて活動を行なってきた.この活動を通して,メンバ全員が全ての工
程に関わることで一つの工程に特化することなく,どんな分野でも活躍できるスキルを身につけ
ることができた.要求定義,要件定義などのアプリケーションの仕様を決定する段階では,自分た
ちの考えを正確に伝えることや,お互いの意見の食い違いや矛盾を解消することや,相手の意見を理解し
協調するような心構えが必要となった.この経験から,メンバ全員が話合いにおいて主体性の重要
性を学んだ.また,今年の特徴としては,未来大がアプリケーション「Cool japanimation」と「Rhythwalk」のビジ
ネスモデルも担当した.未来大はビジネスモデルに関する知識はなかったので,専大の速水先生からメールでアポイントをとり,アドバイ
スを頂いたり,これまでのミライケータイプロジェクトのビジネスモデルを参考にするなどして,
文系の分野についても学ぶことができた.これにより,文理融合が実現した.各工程で作成するド
キュメントは,学生と教員によるレビューを複数回行い,その後の活動の基盤になるものをしっ
かりと作っていくことができた.
\par
2つ目は,会議の円滑な進め方について学んだことである.あらかじめ会議が始まる前に,その
日の議題についてプロジェクトメンバ全員が把握し,それぞれ考えてくるべきことを明らかにす
ることで,会議を効率的に進めることができた.また,会議中の活動として学んだことは,議事録
をしっかりとることの重要性である.議事録をとることによって,会議の流れをうまく操作し,ス
ムーズな進行をすることができるようになった.また,その後の活動で,分からないことや確認し
たいことがあるときに,議事録を見ることで何があったのかすぐに把握することができるように
なった.議事録は,メンバ全員が担当できるように,毎回議事録をとる担当を変え,全員が議事録
をとるための技術を身につけられるように工夫した.会議では,主に他大学との進捗確認や、連絡事項の確認を
行い,大学ごとに情報量の差がないように注意したながら、情報共有をした.
\par
3つ目は,実践を通してアプリケーション開発技術を学んだことである.今年の特徴としては,
Android班が両アプリケーションを開発を担当したことである.開発は,Android,iOS,HTML5 の開発を行い,
さらに,他大学と連携し共同開発を行った.他大学との連携によって,開発に様々な不具合が
発生したが,問題は一人で抱えるのではなく,メンバ全員で問題を共有し,それを解決することによって開発技術力を向上させることができた.
開発を通して根気よくプログラムを書き進めていく粘り強さを身につけることができた.
\par
4 つ目は,コンペティションの書類審査やアプリケーションの仕様書を作成することを通して仕
様書やドキュメント等の作成知識を身につけたことである.私たちは,キャンパスベンチャーグラ
ンプリ1 次審査のための書類作成やそれぞれのアプリケーション仕様書作成を通して,文章や図で
正確にアプリケーションの仕様について伝えられるように記述する技術を身につけた.また,仕様
書ごとに矛盾があることの無いように,アプリケーションの仕様について統一性を持たせることに
注意した.ドキュメント作成に関して,メンバで何度も推敲し,先生方にもレビューを頂き,文章
の誤りや誤解を生まないように作成することを心がけた.
\par
5 つ目は,イベントや発表会に積極的に参加し,発表スキルを身につけたことである.私たちは,
未来大でのオープンキャンパス,中間発表会,成果報告会以外にも,アカデミックリンクに参加し,多くの発表を経験してきた.
メンバ全員が発表に参加できるよう調整し,経験することで,発表技術がプロジェクト前と比べると格段に向
上した.発表資料となるスライドやポスター作りにおいても,メンバ同士で何度も推敲し,発表練
習でもお互いにアドバイスをするなどして納得のいくものを作れるよう努力した.これらから,聴
衆にわかりやすく説明するための考え方や,それらを形にして実行する技術を身につけた.
\bunseki{金澤 しほり(未来大)}
