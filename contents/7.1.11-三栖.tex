\subsection{三栖 惇}
\par
5月には合同合宿に向けたアイディア出しやその合宿に向けたデモアプリケーションの開発をiOSデバイスにより行った.6月では第一回合同合宿を行い,開発するアプリケーションを決定した.さらに仕様書の作成にも取り掛かり,要求定義書および要件定義書のリーダとして二つの仕様書を作成した.7月は要求定義書,要件定義書を書くべく会議を進めると共にサービス仕様書,詳細仕様書の作成にも入り,加えて中間発表の準備にも取り掛かった.中間発表では,メインポスターの作成や発表スライドとアプリケーションのポスターの作成協力も行った.またここでもデモアプリケーションを作成し,中間発表会に臨んだ.8月に入ってすぐオープンキャンパスでポスターセッションを行った.そして,夏休みである8月,9月ではアプリケーション開発に必要な技術習得やアプリケーションの開発,後半に入ってはビジネスモデルの考案などを行った.10月,初旬はアプリケーションの開発を少し行ったが,中旬以降は「Rhyth/Walk」のキャンパスベンチャーグランプリのリーダとして,ビジネスモデルの考案に努めた.下旬にはそのその応募と選考書類を提出した.11月に入っては,アカデミックリンクにポスターとデモアプリケーションを出したり,第二回合同合宿を行った.また,翌月の最終発表に向けて,ポスタの作成とスライドの作成協力を行った.12月では,最終発表のためのデモアプリケーションを作成し,最終発表に臨んだ.そしてそれが終わると最終報告書の作成を開始した.1月は引き続き最終報告書の作成を行うと共に,2月の企業報告の準備に取り掛かる.
\par
5月
\begin{itemize}
\item 合宿に向けたアイディア出し
\item 第一回合同合宿用デモ作成(iOS)
\end{itemize}
6月
\begin{itemize}
\item 第一回合同合宿
\item 要求・要件定義書リーダに就任
\item 要求定義書の作成
\item 要件定義書の作成
\end{itemize}
7月
\begin{itemize}
\item 要求定義書のための会議に参加
\item 要件定義書のための会議に参加
\item サービス仕様書のための会議に参加
\item 中間発表用のメインポスターの作成
\item 中間発表用の「Rhyth/Walk」のスライド作成協力
\item 中間発表用の「Rhyth/Walk」のポスターを作成協力
\item 中間発表第 1 次「Rhyth/Walk」の台本の作成協力
\item 中間発表用デモ作成(iOS)
\item 中間発表
\item 詳細仕様書の作成
\end{itemize}
8月
\begin{itemize}
\item オープンキャンパスの用意
\item 必要な技術の習得
\item アプリケーション開発
\end{itemize}
9月
\begin{itemize}
\item 必要な技術を習得
\item アプリケーション開発
\item ビジネスモデルのための調査
\end{itemize}
10月
\begin{itemize}
\item 必要な技術を習得
\item アプリケーション開発
\item キャンパスベンチャーグランプリ「Rhyth/Walk」班リーダに就任
\item キャンパスベンチャグランプリへの書類作成
\item キャンパスベンチャグランプリへの応募
\end{itemize}
11月
\begin{itemize}
\item アカデミックリンク
\item 第二回合同合宿
\item 最終発表用の「Rhyth/Walk」のポスターを作成
\item 最終発表用の「Rhyth/Walk」のスライド作成協力
\end{itemize}
12月
\begin{itemize}
\item 最終発表用デモ作成(iOS)
\item プロジェクト最終報告会
\item 最終報告書の作成
\end{itemize}
1月
\begin{itemize}
\item 最終報告書の作成
\end{itemize}
2月
\begin{itemize}
\item 企業報告会の準備
\item 企業報告会
\end{itemize}
\par
今年からiOS開発はSwiftという新言語を導入した.今年できた言語ということもあってリファレンスの数が少なく,苦労することも多かったが,少ない記事や英語の記事から有用な情報を引っ張ってくるなどの経験ができた.また,新しく言語を習得することにも慣れることができ,今後この知識と経験を役立てていきたい.
\par
キャンパスベンチャーグランプリのリーダとして,ビジネスモデル,収益モデルを考えたことは大きな経験となった.右も左も分からない状態から,過去の事例や現在の流行,未来の考察といった今までに考えたことのないことを考えさせられた.私自身がそうであったために,同じ班員をうまく牽引できたかどうかわからないが,一緒に悩み,苦労した経験はこれから必ず糧となるものであると思う.
\bunseki{三栖 惇(未来大)}
