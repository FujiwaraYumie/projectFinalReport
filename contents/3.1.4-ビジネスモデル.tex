\subsection{ビジネスモデル説明}
\par
 ここでは,ビジネスモデルについて述べる.本ビジネスモデルは,大きく分けて3つの柱によって構成されている.それをいかに示す.
\begin{enumerate}
\item アプリケーション内のバナー広告によるクリック型広告収入
\par 
本アプリケーション内でバナー広告をだし,ユーザにクリックしてもらうことにより,ワンクリック数円の広告収入を得る.また,バナー広告をアニメ関連のものにすることにより,ユーザにさらなる本アプリケーションの使用意欲をかきたてさせるとともに,クリック数を増やす.
\item 有料コンテンツによる追加課金 
\par
 このアプリケーションには,アニメ検索という機能があり,課金することで,検索したアニメをお気に入りとして登録できるお気に入り機能が使えるようになったり,チャット機能で,このアプリケーション限定で使えるスタンプなどが使えたりできるようになる.
\item 提携店舗とのクーポンによる広告収入
\par
 訪日を終えた後に,航空券などの日本に行ったことを証明できるものをアプリケーションに提示すると,提携店舗で使用できるクーポン券が発行される.そのクーポンは次回訪日する際に,ユーザは使用できる.また,提携店舗は,アニメイトなどのアニメ関連店舗や,イベントや聖地付近の宿泊施設である.また,クーポンとは,アニメ関連店舗であれば,限定グッツがもらえるものであったり,宿泊施設などでは,割引券などのことである.このクーポンを広告料として,提携店舗から収入を得る
\par
以上が,12月段階でのビジネスモデル案の説明である.
\end{enumerate}
\bunseki{紺井 和人(未来大)}
