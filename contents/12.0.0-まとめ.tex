\par
本プロジェクトでは,未来大,神奈工,長崎大の3大学が連携してアプリケーションの企画から
設計,開発を行う実践的な開発プロセスを学ぶことを目的として,スマートフォンアプリケーションの開発を行った.
また,本プロジェクトでは以下の企業と,合同合宿の場でメンバとともにレビューをしたり,
実機提供などという形で協力していた.
\begin{itemize}
\item 日本ヒューレット・パッカード株式会社
\item Y!mobile株式会社
\item 株式会社エヌ・ティ・ティ・ドコモ
\item ソフトバンクモバイル株式会社
\item 株式会社サイバー創研
\item KDDI 株式会社
\item 株式会社NTC
\item IDY 株式会社
\end{itemize}

プロジェクトの前期の活動としては,アプリケーションの企画と設計を主な活動として行い,平行してビジネスモデルの考案も行った.
はじめに各大学でアプリケーションのアイディアを提案し,それらのアイディアを元に
第一回合同合宿でCool JapanimationとRhyth/Walkという2つのアプリケーションを開発することが決まった.
開発の分担は未来大と長崎大がCool JapanimationをAndroidとHTML5で行い,
未来大と神奈工がRhyth/WalkをAndroidとiOSで行うこととした.

その後は各アプリケーションについて開発するにあたって必要となる4種類の仕様書の作成を行い,
アプリケーションの詳細を決定した.
\par
後期は,実際に開発に着手するとともに,平行してアプリケーションのビジネスモデルについても考案を進めた.
今後は企業報告会に向けてアプリケーションの開発を継続するとともに,ドキュメントなどのプロジェクトの成果物まとめていく.
\bunseki{小笠原 佑樹(未来大)}
