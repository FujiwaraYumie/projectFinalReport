\subsection{神奈川工科大学の役割}
\par
今回,本プロジェクトに参加している神奈工の学生は3名であり,全員が情報メディア学科というプログラミングについての勉学を主とした学科に所属している.
\par
アプリケーションの開発では,Android版「Rhyth/Walk」の開発を担当する.
企画から開発までの工程や,他大学との連携プロジェクトの学習という目標の元,今回もこれまでの神奈工と同じように,各仕様書からビジネスモデルの提案まで,プロジェクトのほぼすべてに携わる.
\par
情報メディア学科は,様々なプログラミング言語の学習以外にも,CG,ミュージック等,コンテンツ分野に関して幅広い勉学を行っている. そのため, 本プロジェクトにおいて,アプリケーションの開発やアイディアの提案などの貢献が可能であり,それを神奈工の役割としている.
\bunseki{赤木 詠滋(神奈工)}
