\section{背景}
\par 
現在,現在スマートフォンの普及率は増加しており,世代問わずにスマートフォンが持っていることが当たり前の時代となった.
日本でのクライアント端末別の普及率はスマートフォンが49.8%,タブレット端末が20.1%,PCが97.0%だった.
年齢別では,スマートフォンは若年層ほど所有率が高く,高齢者層は従来型携帯電話(フィーチャーフォン)の所有率が高かった.
一方,タブレットは40代が所有率のピーク(21.6%)だった.
サービス提供では,クラウドコンピューティングというコンピューティング形態が使用されており,現在世界的に普及が進んでいる.
クラウドコンピューティングとは,サーバがユーザに提供するサービスをサーバ群を意識せずに利用が可能になるコンピューテング形態のことである.
したがってサービスやアプリケーションの可能性は年々広がってきている.
\par
具体的な例として,電子メール,カメラ機能インターネット利用,アプリケーション機能,ミュージックプレイヤー機能,GPS機能,Bluetooth,Felica等のICカードを利用した非接触機能,赤外線等の無線通信機能等が挙げられる.
これらの機能により携帯電話は,もはや通信機器という域を超え,日常生活の様々な場面で必要不可欠なものになっている.
時には,いつでもどこでも知人と友人と会話できるコミュニケーションツールとして,またある時は子供を犯罪から守る防犯ツール,更には財布がなくても精算できるおサイフケータイなど,携帯電話は幅広く使用されている.
こうした様々な用途がある機能の中から今回のプロジェクトではアプリケーション機能について注目する.
そして,携帯電話にアプリケーション機能が搭載されていることは標準となり,それに伴って,docomo,WILLCOMなどの各社から様々な機能やサービスが提供されている.
また,iOS,Android,HTML5などのプラットフォームが普及し始めたことにより,さらに幅広いサービスの利用も可能になっている.
以上のような背景を踏まえ,本プロジェクトでは既存のアプリケーションにとらわれない新しい発想でスマートフォンアプリケーションの提案と開発を行う.
\bunseki{坂本 豊教(未来大)}
