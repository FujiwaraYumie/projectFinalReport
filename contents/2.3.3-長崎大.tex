\subsection{長崎大学}
\begin{enumerate}
\item 開発アプリケーションのアイディア提案 
\par 長崎大としてもともと研究対象となっていた LOD を用いた観光支援アプリケーションを第一回合宿の長崎大の案として決定した.
\item 開発アプリケーションの決定 
\par 第一回合宿で各校が持ち寄ったアプリケーション案をそれぞれプレゼンし、厳選した後それぞれのアプリに取り組むメンバーを混合してよりブラッシュアップした案の中から開発アプリケーション2つを決定した. 
\item 開発アプリケーションに関する調査 
\par  アプリケーションに携わるメンバーを確定した後,このアプリケーションのターゲットや背景、開発プラットフォームについて、メンバーでの話し合いによって決定した.
\item アプリケーションの機能設定 
\par 合同会議などでアイディアとして出てきた機能の中から取捨選択,または追加し,実装する機能を決定した. 
\item アプリケーション名の決定 
\par 第一回合同合宿時の開発アプリケーションの決定をする際に使用したアプリケーション名をそのまま用いることに決定した.
\item ビジネスモデル作成技術の習得 
\par 長崎大のメンバーで話し合いひとつのビジネスモデルを作成・ブラッシュアップし,より詳しいビジネスモデルに仕上げた.
\item 要求定義書の作成 
\par 開発アプリケーションがどのような仕様を設けるかを記したものを、2 校で項目ごとに分担して合同で1つの要求定義書を作成した. 
\item 要件定義書の作成 
\par アプリケーションのソフトウェア要件をまとめたものを、2 校で項目ごとに分担して合同で1つの要件定義書を作成した.. 
\item サービス仕様書の作成 
\par アプリケーションがユーザ視点でどのようなサービスを提供するかを記したものを、2 校で項目ごとに分担して合同で1つのサービス仕様書を作成した.
\item 前期提出物の作成 
\par 2校で項目ごとに分担して,2校合同で中間報告書を作成した. 
\end{enumerate}
\bunseki{立石 拓也(長崎大)}
