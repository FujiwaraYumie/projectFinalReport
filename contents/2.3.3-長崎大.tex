\subsection{長崎大学}
\begin{enumerate}
\item 開発アプリケーションのアイディア提案 
\par 長崎大としてもともと研究対象となっていた LOD を用いた観光支援アプリケーションを第一回合宿の長崎大の案として決定した.
\item 開発アプリケーションの決定 
\par 第一回合宿で各校が持ち寄ったアプリケーション案をそれぞれプレゼンし,厳選した後それぞれのアプリに取り組むメンバを混合してよりブラッシュアップした案の中から開発アプリケーション2つを決定した. 
\item 開発アプリケーションに関する調査 
\par  アプリケーションに携わるメンバーを確定した後,このアプリケーションのターゲットや背景,開発プラットフォームについて,メンバでの話し合いによって決定した.
\item アプリケーションの機能設定 
\par 合同会議などでアイディアとして出てきた機能の中から取捨選択,または追加し,実装する機能を決定した. 
\item アプリケーション名の決定 
\par 第一回合同合宿時の開発アプリケーションの決定をする際に使用したアプリケーション名をそのまま用いることに決定した.
\item ビジネスモデル作成技術の習得 
\par 長崎大のメンバで話し合いひとつのビジネスモデルを作成・ブラッシュアップし,より詳しいビジネスモデルに仕上げた.
\item 要求定義書の作成 
\par 開発アプリケーションがどのような仕様を設けるかを記したものを,長崎大・未来大の2 校で項目ごとに分担して合同で1つの要求定義書を作成した. 
\item 要件定義書の作成 
\par アプリケーションのソフトウェア要件をまとめたものを,長崎大・未来大の2 校で項目ごとに分担して合同で1つの要件定義書を作成した. 
\item サービス仕様書の作成 
\par アプリケーションがユーザ視点でどのようなサービスを提供するかを記したものを,長崎大・未来大の2 校で項目ごとに分担して合同で1つのサービス仕様書を作成した.
\item 前期提出物の作成 
\par 長崎大・未来大の2校で項目ごとに分担して,2校合同で中間報告書を作成した.
\end{enumerate}
\bunseki{立石 拓也(長崎大)}
