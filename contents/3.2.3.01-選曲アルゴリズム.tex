【選曲アルゴリズム】
\par
%選曲アルゴリズム機能は,手持ちの楽曲をBPM解析とスケール解析,歌詞解析で解析したデータと
%歩くテンポや時間,季節,場所,天気から選択した優先度の2つをもとに,再生する楽曲を選曲する機能である.

%\par
%解析された楽曲と使用者が設定している優先度が一致したとき,一致度が高い順に楽曲を選曲する.
%選曲の例として,12月に季節の優先度を0~1の中から1に設定されていたとする.まず,12月から
%季節が冬だと判断する.次に歌詞から冬に関連するキーワード,例えば冬や雪という単語を
%カウントする.このカウントしたキーワードの合計に優先度の数値を掛けることで,重み付けを行う.
%最後に,重み付けを行った結果をもとに,数値の高い順に楽曲を選曲する.BPMを除いた他の優先度も同様に選曲する.

%\par
%BPMの優先度をONにした場合,自動で取得した使用者のBPMと最も一致する楽曲から順に楽曲を選曲する.
%より具体的には,使用者のBPMの数値から楽曲のBPMの数値を引く.この結果に絶対値を掛け,
%数値が0に近い順に楽曲を選曲する.

%\par
%複数の優先度が設定されていた場合,設定された優先度を全て考慮し,選曲を行う.
%複数の例として,季節と時間の優先度が設定されていたとする.各優先度毎に,
%先程の季節の例と同様に重み付けを行う.優先度設定で考慮する季節,
%時間の値を合計し,数値が高い順に選曲する.
