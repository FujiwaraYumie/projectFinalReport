【選曲アルゴリズム】
\par
選曲アルゴリズム機能は,手持ちの音楽をBPM解析とスケール解析,歌詞解析を用いて音楽を解析した情報と
ユーザ周辺の歩くテンポや時間,季節,場所,天気から取得した情報を複合的に解析,
また取得する歩くテンポや時間,季節,場所,天気に対し,優先度を設定することで選曲する機能である.

\par
解析された音楽と使用者が設定している優先度が一致したとき,一致度が高い順に音楽を選曲する.
選曲の例として,12月に季節の優先度を0~1の中から1に設定されていたとする.まず,12月から
季節が冬だと判断する.次に歌詞から冬に関連するキーワード,例えば冬や雪という単語を
カウントする.このカウントしたキーワードの合計に優先度の数値を掛けることで,重み付けを行う.
また,自動で取得したユーザのBPMと最も一致する音楽のBPMの数値から音楽のBPMの数値を引く.
この結果から絶対値をとり,数値が0に近い順に音楽に重み付けを行う.これらの重み付けされた
値を複合的に考慮することで,選曲を行う.

\bunseki{木津 智博(未来大)}
