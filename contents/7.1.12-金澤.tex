\par
活動内容と予定
\par
5月はプロジェクトで開発するアプリケーション案を考案し、海外やブラッシュアップでアプリケーション案を絞込み、
合宿に向けてスライド作成を行った.
6月には第一回合宿を行い,アプリケーションが決定した.
開発のプラットフォームはHTML5になったのと同時にHTML5班のリーダに就任した.
7月は学内で中間発表会が行なわれた.
中間報告書を書いて提出するため,グループ報告書のリーダに就任した.
8,9月は「Cool Japanimation」のメンバのみで会議を開き,開発する機能の分担をした.
10月はキャンパスベンチャーグランプリに参加するためコンペティション書類作成を行った.
11月はアカデミックリンクに参加するため,デモの開発を行った.
12月は成果発表会があったので,発表練習とレビューを重ねた.
最終報告書の提出に向けて,再びグループ報告書リーダに就任した.
1月で作成したアプリケーションのテストを行う.
また,2月には本プロジェクトの最終ゴールである,企業報告会がある.これからは,企業報告会に向け,スライド
や,ポスターなどの発表資料の改良や,アプリケーション自体の改良を行なっていく予定である.
\subsection{金澤 しほり}
5月
\begin{itemize}
\item アプリケーション案の発案
\item 会議やブラッシュアップでのアプリケーション案の絞り込み
\item グループで会議,スライド作成
\item Skype会議
\item  他大学や未来大内で話し合いを行う必要があるため,Skypeを使用した会議を行った.
\end{itemize}
6月
\begin{itemize}
\item 第一回合同合宿を実施
\item 用意した案の発表で質疑応答
\item グループでのcool japanimationの検討
\item 合宿で決定した案を元に要求定義書,要件定義書,サービス仕様書の作成
\item HTML5技術習得リーダ就任
\item Skype会議
\item  他大学や未来大内で話し合いを行う必要があるため,Skypeを使用した会議を行った.
\end{itemize}
7月
\begin{itemize}
\item HTML5技術習得
\item Skype会議
\item  他大学や未来大内で話し合いを行う必要があるため,Skypeを使用した会議を行った.
\item 中間発表会
\item 学内でプロジェクトの中間発表会が行なわれた.
\item 中間報告書について
\item   中間報告書のリーダー担当になった.
\item   中間報告書の担当割り振りや,スケジュール作成を行った.
\item   そして自身も中間報告書の作成を行った.
\item サービス仕様書
\item 詳細仕様書の作成
\end{itemize}
8月
\begin{itemize}
\item Cool Japanimation-LINE会議
\item   他大学や未来大内で話し合いを行う必要があるため,LINEを使用した会議を行った.
\item 機能担当分担
\item  会議の際に,各機能の開発の担当決めを行った.
\item レビュー機能開発
\item   レビュー機能の担当になったので,開発を行った.
\end{itemize}
9月
\begin{itemize}
\item レビュー機能開発
\item  レビュー機能の担当になったので,開発を行った.
\end{itemize}
10月
\begin{itemize}
\item Cool Japanimation-LINE会議
\item   他大学や未来大内で話し合いを行う必要があるため,LINEを使用した会議を行った.
\item Skype 会議
\item 他大学や未来大内で話し合いを行う必要があるため,Skypeを使用した会議を行った.
\item レビュー機能開発
\item   レビュー機能の担当になったので,開発を行った.
\item 機能のマージ作業
\item   各機能が単体で出来上がったので,その機能をマージして一つのアプリケーションで使用できるようにマージ作業を行った.
\item キャンパスベンチャーグランプリ
\item   キャンバスベンチャーグランプリにエントリーするため,
\item   アプリケーションの説明をドキュメント化し,ビジネスモデルの考案を行った.
\end{itemize}
11月
\begin{itemize}
\item Cool Japanimation-LINE会議
\item  他大学や未来大内で話し合いを行う必要があるため,LINEを使用した会議を行った.
\item Skype会議
\item   他大学や未来大内で話し合いを行う必要があるため,Skypeを使用した会議を行った.
\item 機能のマージ作業
\item   各機能が単体で出来上がったので,その機能をマージして一つのアプリケーションで使用できるようにマージ作業を行った.
\item アカデミックリンク
\item   アカデミックリンクに参加し,アプリケーションのデモが行えるようにデモ用に作成した.
\item 第二回合同合宿
\item   第二回合同合宿が未来大で行なわれた.合宿用にCool Japanimationの進捗スライドの作成を行った.
\end{itemize}

12月
\begin{itemize}
\item Skype 会議
\item   他大学や未来大内で話し合いを行う必要があるため,Skypeを使用した会議を行った.
\item ツアー参加申請機能提案
\item    ツアー参加申請を新たに付け足すことにしたため,画面遷移図など考えた. 
\item ツアー参加申請機能開発
\item    ツアー参加申請機能の画面遷移図なを元に開発を行った.
\item プロジェクト成果発表会
\item    学内でプロジェクト成果発表会が行なわれた.
\item    そのため,プロジェクト成果発表用のデモ作成を行った.
\item 最終報告書
\item   最終報告書リーダーになった.
\item   そのため,最終報告書のスケジュール作成に,三大学分の担当割り振りを行い,
\item   最終報告書作成を行った.
\end{itemize}

1月
\begin{itemize}
\item Skype 会議
\item   他大学や未来大内で話し合いを行う必要があるため,Skypeを使用した会議を行った.
\item Cool Japanimation-LINE会議
\item   他大学や未来大内で話し合いを行う必要があるため,LINEを使用した会議を行った.
\item ツアー参加申請機能開発
\item   ツアー参加申請機能の画面遷移図なを元に開発を行った.	
\item 機能のマージ作業
\item   各機能が単体で出来上がったので,その機能をマージして一つのアプリケーションで使用できるようにマージ作業を行った.
\item 最終報告書の作成をした
\item 企業報告会の準備をした
\end{itemize}

2月
\begin{itemize}
\item ツアー参加申請機能開発
\item 機能のマージ作業
\item Cool Japanimation-LINE会議
\item   他大学や未来大内で話し合いを行う必要があるため,LINEを使用した会議を行う.
\item Skype 会議
\item   他大学や未来大内で話し合いを行う必要があるため,Skypeを使用した会議を行った.
\item 企業報告会の準備
\item 企業報告会
\end{itemize}

\par 
活動内容と予定
\par
私は本プロジェクトのアプリケーション開発で,「Cool Japanimation」を担当することとなった.
HTML5班に所属し,HTML5の開発班のリーダーを務めた.
開発環境はmonacaをインストールし開発を行った.
言語はHTMLとjavascriptであった.HTMLは以前にも授業で触れたことがあったが,
javascriptは初めてであったので,具体的に自分の担当した機能の設計詳細を考え,
Web 上のサンプルプログラムと解説が乗っているページを見つけて勉強するようにした.
夏休みのときに割り振られた自分の担当した機能は無事完成を迎えることができた.
後期で機能のマージ作業に入ったが,原因がわからず一部機能が動かず
途中でどうしてもやり方が分からない処理もあったが,
その問題をミライケータイプロジェクトのメンバに相談することで解決することができた.
日々の活動の中では,メンバ間で交代し議事録を取った.
わかりやすい議事録を取るのを心がけ,細かいと
ころまで記述するようにした.
また,中間と最終のグループ報告書のリーダーも担当し,
決められた条件を達するように他大学の担当割り振りをバランスよくなるように行い,
期日までに良いものを仕上げるためにスケジュールの管理も心がけた.
\bunseki{金澤 しほり(未来大)}
