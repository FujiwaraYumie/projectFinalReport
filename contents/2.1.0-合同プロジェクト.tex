\section{合同プロジェクト}
本合同プロジェクトは,未来大のシステム情報科学実習のミライケータイプロジェクト・神奈工の速水教授主催のケータイプロジェクト・長崎大の創成プロジェクトによる合同プロジェクトである.
\par
本合同プロジェクトの目標は,「携帯電話などのモバイル端末を有効利用し,クラウド技術やスマートフォンの各種センサなど最新の技術を活用する携帯電話プラットフォーム間をシームレスにつなぐモバイルマルチメディアサービスの提案と開発の学習,ビジネスモデルの提案と学習」というテーマの下,異なった分野の大学同士の連携によるシステム提案,それに基づくシステムの開発とシステムの評価,ビジネスモデルの構築をお互いの得意分野を活かしつつ,それぞれの分野からの視点を交えて行うことである.
\par
携帯電話などのモバイル端末を有効利用し,クラウド技術や携帯電話の各種センサなど最新の技術を活用する携帯電話プラットフォーム間をシームレスにつなぐモバイルマルチメディアサービスの提案と開発の学習,ビジネスモデルの提案の学習にあたり,私たちはCool Japanimation,Rhyth/Walkという二つのアプリケーションを開発する.
\par
Cool Japanimationではキャリアはau,docomo,SoftBank,Y!mobile,プラットフォームはAndroid,HTML5で同一のサービスを受けられることを前提とする.同様に,
Rhyth/Walkではキャリアはau,docomo,SoftBank,Y!mobile,プラットフォームはAndroid,iOSで同一のサービスを受けられることを前提とする.
\par
これらのアプリケーションでは,将来性があり,楽しく便利なサービスを提供する.また,本プロジェクトでは,アプリケーションの開発を通して,ソフトウェア開発の学習,ビジネスモデルの学習,本プロジェクトを終えて何を得るのかを考えること,企画と開発に分かれ,しっかりと開発プロセスを学ぶことが目標である.これらの目標を十分に満たすために,本プロジェクトではそれぞれの大学でプラットフォームの調査やメンバ個々でのアイディアの提案, グループでの話し合いによるブレインストーミングやアイディアの統合,プレゼンテーションを繰り返し行い,上述したCool Japanimation,Rhyth/Walkという二つのアプリケーションを開発することとなった.
\bunseki{木津 智博(未来大)}
