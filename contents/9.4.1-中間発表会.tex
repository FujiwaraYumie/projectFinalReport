\subsection{中間報告会}
2014年11月14日,長崎大学にて成果発表会に向けて成果を発表する,中間報告会が行われた.
それに向けて,11月上旬から発表するアプリケーションCool Japanimationの発表用のスライドをそれぞれ作成し始めた.
以下に,中間報告会で用いた各種発表資料の準備内容と役割分担を示す.
 
 Cool Japanimation 

スライド(アプリの説明・サーバ:磯野,アカウント:岡本,ナビ:吉澤) 

本プロジェクトにおいての,アプリ案の洗い出しやコンセプト,ターゲット,アプリケーションの説明,ビジネスモデル,現在の進捗状況,今後の予定,そして画面遷移が行われている映像をスライドにしたものである.
アプリの概要説明のスライドは,第一回合同合宿のアプリを説明する際に使ったスライドを元に製作を行い,各機能の説明はその機能の担当者が制作を行った.
発表を聞いていただいたのがアプリケーション製作は専門外という方々が多かったため,機能の技術を説明するのは最小限にして,アプリケーションの特徴を説明することを重視した.
また,今回の中間報告会が比較的大人数に向けたものであったのと,時間が10分と限られていたため,実際に実機に触れていただくデモンストレーションを行うのは非効率であると考えた.
そこで,実際に画面遷移が行われている映像を見せながら説明を行うようにすることにより,より多くの人にアプリの特徴を知ってもらうことができた.

発表者(長崎大:磯野,岡本,吉澤)

作成されたスライドを用いて聴衆の前で中間報告会を行った.
報告会では私達の他に3つのチームが発表した.
1チームの発表時間は10分とされ,その後質疑応答の時間が設けられた.
発表者は担当のスライドの内容を把握し,損門的分野ではなくてもわかるような発表を心がけた.
中間報告前にスライドをメンバー全員で確認し,意見を出しあった後,先生にも見ていただき,より良い報告ができるように務めた.
当日は多くの来場者に本プロジェクトの活動と成果を伝えることができた. 

質疑応答(長崎大:磯野) 

質疑応答の時間では,15分ほどの時間が取られ,他のチームのメンバーや,偏った分野ではなく,様々な分野の先生方から発表やアプリの機能に対する意見を頂いた. 

全体評価 

中間発表の聴講者による質疑応答から自分のグループの評価を行うと,発表技術についてはスライドに関して,発表順序を変えたほうがいいのではと言った意見や,聖地や巡礼といった専門的な用語の説明が不十分ではないかという意見を頂いた.
アプリに関しては,もっとこの角度から写真を撮れる同じ場面が撮れるといった細かい要望や,果たしてこの機能をアプリでするメリットとは何なのか,といった厳しい意見を頂いた.
偏った分野ではなく,様々な分野の先生方から意見をいただくことができて,私達では思いつきもしなかったアイデアや問題があり,とても有意義な時間となった.

吉澤健太(長崎大)
