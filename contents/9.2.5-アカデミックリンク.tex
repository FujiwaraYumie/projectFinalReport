\subsection{アカデミックリンク}
\par
ここでは,2014年11月8日(土)に行なわれたアカデミックリンクについて述べる.
はこだて高等教育機関合同研究発表会である函館アカデミックリンク2014に参加した.
アカデミックリンクは函館市内にある8つの大学・短大・高専で,学生・教員らが
持つ「体験」「発想」「探求心」から生まれる様々な研究を函館市民の皆様・地元企業の方々
にわかりやすく発表し,また各研究テーマの協力・連携の可能性を探るべく一同に結集するイベントである.
私たちはこのイベントでブースセッションの発表形式で出場した.
以下に,活動について記述する.
\par
ブースセッションの準備として,ポスターと開発アプリケーションのプロトタイプをいれた実機を用意した.
実際に実機に触ってもらいながら操作しながら説明を行うことが一番理解しやすく,
聞き手の興味を引くだろうと思い,アプリケーションに触れてもらう形を採用した.
以下がブースセッションの際に用いたものである.

\\
・メインポスターはタイトル,メンバ名,プロジェクト概要,スケジュール,活動,協力企業から構成されている.
また,教員からレビューをもらい,何度も手直しを行い,来訪者に簡単に理解してもらうため見やすくした.
\\
・アプリケーション「Cool Japanimation」\\
ポスターはタイトル,概要,コンセプト,ターゲット,利用例から構成されている.
プロトタイプはアカデミックリンクの時点までに開発していたアプリケーションを実機にダウンロードし,展示した.
「Cool Japanimation」はアプリケーション内のアニメ検索機能と,行きたい既存ツアーを検索する機能,新規ツアーを企画する機能,
ツアーに対してのレビューをする機能などを実装しているプロトタイプを展示した.
\\
・アプリケーション「Rhyth/Walk」\\
ポスターはタイトル,概要,ターゲット,アプリの流れから構成されている.
プロトタイプはアカデミックリンクの時点までに開発していたアプリケーションを実機にダウンロードし,展示した.
「Rhyth/Walk」は端末内音楽を再生する機能と,現在のシチュエーション(天気,時間,季節,BPM)を取得する機能,
ユーザの歩いているテンポを取得して選曲する機能などを実装しているプロトタイプを展示した.
\\
<結果・反省>
\par
アプリケーションのターゲットユーザである学生や一般の方々の視点からアイディアをいただくことができ,
アプリケーションのサービス向上につながった.
ブースが1つでは狭く感じたり,プロジェクトメンバがブースに何人もいたので来訪者は近寄りづらい環境だったため,
ブース内にいるのは2人程度が妥当であると反省した.
また,2つのアプリケーションを開発したので2つのブースを借りるべきだった.
周りはポスタが多く,差別化や良い意味で目立たせるためにもモニタを使ったアプローチ方法が良かったのではないかと考えた.
本プロジェクト内で2つのアプリケーションを作っているが,担当ではないアプリケーションについて来訪者に質問をされることもあったが,
情報共有が十分でなかったため,うまく対応することができなかった.

\bunseki{坂本 豊教(未来大)}\par
